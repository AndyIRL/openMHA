%\documentclass[11pt]{beamer}
%
%\usepackage[english]{babel}
%\usepackage{hyperref} 
%%\hypersetup{pdfpagemode=FullScreen} 
%\usepackage[utf8]{inputenc}
%% Font
%\usefonttheme{default}
%\usepackage{lmodern}
%\usepackage[T1]{fontenc}
%

%
%\newcommand{\argmax}[1]{\underset{#1}{\operatorname{argmax}}}
%
%\usetheme{medi-de}
%\setbeamersize{text margin left=5mm, text margin right=5mm}
%

%%%%%%% footnotes for chapter
%\usepackage{manyfoot}
%\newfootnote{B}
%\makeatletter
%\def\sfootnote#1{\ifx\protect\@typeset@protect
    %\Footnotemark{\small*}\FootnotetextB{*}{#1}%
  %\else
    %\relax
  %\fi
%}
%
%
%\bibliographystyle{unsrt}
%\title[Binaural Source Localization] % (optional, nur bei langen Titeln noetig)
%{\only{(Binaural) Source Localization with Machine Learning approaches}}
%\author[Hendrik Kayser]{\textbf{Hendrik Kayser}} 
%\date{23.06.2016}


%%%%%%%%%%%%%%%%%%%%%%%%%%%%%%%%%%%%%%%%%%%%%%%%%%%%%%%%%%%%%%%%%%%%%%%%%%%
%%%%%%%%%%%%%%%%%%%%%%%%%%%%%%%%%%%%%%%%%%%%%%%%%%%%%%%%%%%%%%%%%%%%%%%%%%%
\pdfobjcompresslevel=0

\documentclass[10pt,hyperref={pdfpagelabels=false},aspectratio=169]{beamer}
%

\usepackage[ansinew]{inputenc}
\usepackage[T1]{fontenc}
\usepackage{array}

\usepackage{graphics}

\setbeamersize{text margin left=.4cm,text margin right=.4cm} 

\DeclareGraphicsExtensions{.pdf,.png,.jpg}
%%%%%%%%%%%%%%%%%%%%%%%%%%%%%%%%%%%%%%%%%%%%%%%%%%%%%%%%%%%%%%%%%%%%%%%%%%%
% here we're going to use the H4a theme:
%\usetheme[university=UOL]{H4a}
\usetheme[university=UOL]{H4a}
% valid options for 'group=' are MEDI,AKU,SIGPROC,none
% valid options for 'lang=' are en,de
%%%%%%%%%%%%%%%%%%%%%%%%%%%%%%%%%%%%%%%%%%%%%%%%%%%%%%%%%%%%%%%%%%%%%%%%%%%

%\title{\vspace{.7cm}\\The open Master Hearing Aid - openMHA}
%
%%\subtitle[short subtitle] % optional
%%{[long subtitle]}
%
%\author{Hendrik Kayser}
%\institute{%
%Digital hearing devices\\
%Medizinische Physik and Cluster of Excellence Hearing4all \\
%Carl von Ossietzky University Oldenburg }
%
\date{20/06/2017}  % using \today may produce long format, depending on your locale

\begin{document}

%%%%%%%%%%%%%%%%%%%%%%%
%{
%\setbeamertemplate{footline}{} 
%\begin{frame}
  %\titlepage
%\end{frame}
%} % needs to be kept in curly braces 

\begin{frame}{openMHA status}

As of \textbf{June 21st}, the openMHA contains

\begin{itemize}
	\item The basic framework
		\begin{itemize}
			\item MHA host application
			\item libMHA toolbox
			\item communication interfaces to control applications and for audio I/O
		\end{itemize}
			\item Hearing aid processing plugins
		\begin{itemize}
			\item multi-band dynamic compressor
			\item feedback cancellation
			\item binaural coherence filter
			\item adaptive differential microphones 
			\item beamformer (delay-and-sum, MVDR)
			\item single-channel noise suppression
		\end{itemize}
		\item Octave/Matlab GUI for hearing aid fitting 
		\item Documentation for plugin development, application engineering, hearing aid fitting GUI
\end{itemize}


\end{frame}

\begin{frame}{Plans...}
\begin{itemize}
	\item Facilitation of the installation process
	\item Extension of the functionality provided by the fitting GUI (calibration)
	\item Windows/Mac OS support
	\item ARM platform support
	\item Define interface for communication with community - do we need more than GitHub?
	\item Refinement of documentation, taking user questions/feedback into account
	\item Development  of a concept to handle and store client data in fitting GUI more efficiently
	\item Research on new algorithms 
\end{itemize}


\end{frame}



\end{document}