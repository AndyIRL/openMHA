%%% This file is part of the Open HörTech Master Hearing Aid (openMHA)
%%% Copyright © 2019 HörTech gGmbH
%%%
%%% openMHA is free software: you can redistribute it and/or modify
%%% it under the terms of the GNU Affero General Public License as published by
%%% the Free Software Foundation, version 3 of the License.
%%%
%%% openMHA is distributed in the hope that it will be useful,
%%% but WITHOUT ANY WARRANTY; without even the implied warranty of
%%% MERCHANTABILITY or FITNESS FOR A PARTICULAR PURPOSE.  See the
%%% GNU Affero General Public License, version 3 for more details.
%%%
%%% You should have received a copy of the GNU Affero General Public License, 
%%% version 3 along with openMHA.  If not, see <http://www.gnu.org/licenses/>.

% Latex header for doxygen 1.8.11
% adapted for openMHA
\documentclass[11pt,a4paper,twoside]{article}

% Packages required by doxygen
\usepackage{fixltx2e}
\usepackage{calc}
\usepackage{../openMHAdoxygen}
\setlength{\headheight}{13.6pt}
\usepackage[export]{adjustbox} % also loads graphicx
\usepackage{graphicx}
\usepackage[utf8]{inputenc}
\usepackage{makeidx}
\usepackage{multicol}
\usepackage{multirow}
\PassOptionsToPackage{warn}{textcomp}
\usepackage{textcomp}
\usepackage[nointegrals]{wasysym}
\usepackage[table]{xcolor}

% Font selection
\usepackage[T1]{fontenc}
\usepackage{helvet}
\usepackage{courier}
\usepackage{amssymb}
\usepackage{sectsty}
\usepackage{textcomp}
\renewcommand{\familydefault}{\sfdefault}
\allsectionsfont{%
  \fontseries{bc}\selectfont%
  \color{darkgray}%
}
\renewcommand{\DoxyLabelFont}{%
  \fontseries{bc}\selectfont%
  \color{darkgray}%
}
\newcommand{\+}{\discretionary{\mbox{\scriptsize$\hookleftarrow$}}{}{}}

% Headers & footers
\usepackage{fancyhdr}
\pagestyle{fancyplain}
\renewcommand{\sectionmark}[1]{%
  \markright{\thesection\ #1}%
}
\fancyhead[LE]{\fancyplain{}{\bfseries\thepage}}
\fancyhead[CE]{\fancyplain{}{}}
\fancyhead[RE]{\fancyplain{}{\bfseries\leftmark}}
\fancyhead[LO]{\fancyplain{}{\bfseries\rightmark}}
\fancyhead[CO]{\fancyplain{}{}}
\fancyhead[RO]{\fancyplain{}{\bfseries\thepage}}
\fancyfoot[LE]{\fancyplain{}{}}
\fancyfoot[CE]{\fancyplain{}{}}
\fancyfoot[RE]{\fancyplain{}{\bfseries\scriptsize \copyright{} 2019 H\"orTech gGmbH, Oldenburg }}
\fancyfoot[LO]{\fancyplain{}{\bfseries\scriptsize \copyright{} 2019 H\"orTech gGmbH, Oldenburg }}
\fancyfoot[CO]{\fancyplain{}{}}
\fancyfoot[RO]{\fancyplain{}{}}

% Indices & bibliography
\usepackage{natbib}
\usepackage{tocloft}
\setcounter{tocdepth}{2}
\setcounter{secnumdepth}{4}
\addtolength{\cftsubsecnumwidth}{5pt}
\makeindex

% Custom commands
\newcommand{\clearemptydoublepage}{%
  \newpage{\pagestyle{empty}\cleardoublepage}%
}

\usepackage{caption}
\captionsetup{labelsep=space,justification=centering,font={bf},singlelinecheck=off,skip=4pt,position=top}

\setlength\parindent{0pt}
\usepackage{hyperref}
\usepackage[hang,flushmargin]{footmisc}
\usepackage[margin=1in]{geometry}
\usepackage{color}
\usepackage{subcaption}
\usepackage{fancyvrb} %Für Rahmen um Code Boxen

\begin{document}
\begin{comment}
\MHAtitle{Getting Started}
\newpage
\MHAcopyright{}
\newpage
\tableofcontents
\newpage
\end{comment}
\pagenumbering{arabic}


\begin{comment}
windows --interactive mode?

\end{comment}







\section{Dealing with AC Variables}
\label{sec:freqshifter}

%During the openMHA installation process several Matlab functions were copied to the corresponding installation directory. In  the standard case you can find these Matlab function in \textbf{Program Files\textbackslash openMHA\textbackslash mfiles}.
%For this section you will make use of several openMHA Matlab function which you can find in \textit{/usr/lib/openmha/mfiles}. \textbf{Copy} these files to a non-protected directory, e.g. \textit{/home/YourUserName/Documents/}.



\begin{enumerate}
%\item Copy the \textbf{mfiles} folder into a non protected directory, e.g.
%\textbf{\textbackslash Users\textbackslash YourUserName\textbackslash
%Documents}
\item \textbf{End} all running \textbf{mha processes} 
\item \textbf{Open Matlab or Octave} 
\item Set LD\_LIBRARY\_PATH to empty by typing \\ {\ttfamily
  $\rightarrow$ setenv(\textquotesingle
  LD\_LIBRARY\_PATH\textquotesingle,\textquotesingle\textquotesingle)} \\
  into the \textbf{Command Window}
\item Use the Matlab/Octave \textbf{"Current Folder"} control to navigate to:

\begin{itemize}
\item \textcolor{orange}{\textbf{Linux}}: \\
  \textit{/usr/share/openmha/examples/05-frequency-shifting}
\item \textcolor{orange}{\textbf{Windows}}: \\
  \textit{C:\textbackslash Program Files\textbackslash openMHA\textbackslash examples\textbackslash 05-frequency-shifting}
\item \textcolor{orange}{\textbf{macOS}}: \\
  \textit{/usr/local/share/openmha/examples/05-frequency-shifting}
\end{itemize}

\item In order to use the Matlab functions of openMHA type the following using the \textbf{Command Window:} 

\begin{itemize}
\item \textcolor{orange}{\textbf{Linux}}: \\ $\rightarrow$
  {\ttfamily addpath(\textquotesingle{}/usr/lib/openmha/mfiles\textquotesingle{})}
\item \textcolor{orange}{\textbf{Windows}}: \\ $\rightarrow$
  {\ttfamily addpath(\textquotesingle{}C:\textbackslash Program Files\textbackslash openMHA\textbackslash mfiles\textquotesingle{})}
\item \textcolor{orange}{\textbf{macOS}}: \\ $\rightarrow$
  {\ttfamily addpath(\textquotesingle{}/usr/local/lib/openmha/mfiles/\textquotesingle{})}
\end{itemize}


\item Use the Command Window to enable communication with openMHA through java by typing: 

\begin{itemize}
\item \textcolor{orange}{\textbf{Linux}}: \\ $\rightarrow$
  {\ttfamily javaaddpath(\textquotesingle{}/usr/lib/openmha/mfiles/mhactl\_java.jar\textquotesingle{})}
\item \textcolor{orange}{\textbf{Windows}}: \\ $\rightarrow$
  {\ttfamily javaaddpath(\textquotesingle{}C:\textbackslash Program Files\textbackslash openMHA\textbackslash mfiles\textbackslash mhactl\_java.jar\textquotesingle{})}
\item \textcolor{orange}{\textbf{macOS}}: \\ $\rightarrow$
  {\ttfamily javaaddpath(\textquotesingle{}/usr/local/lib/openmha/mfiles/mhactl\_java.jar\textquotesingle{})}
\end{itemize}

\item In order to start a new openMHA instance type \\
  $\rightarrow$ {\ttfamily openmha = mha\_start;}
\item In order to read in the configuration file type: \\
  $\rightarrow$ {\ttfamily mha\_query(openmha,\textquotesingle{}\textquotesingle{},\textquotesingle{}read:coherence\_live.cfg\textquotesingle{});}
\item \textbf{Start JACK Server} using \textbf{JACK Control}\\
  (Setting: Sample Rate = 44100, Frames/Period = 64)
\item Start the mha process by typing \\ $\rightarrow$
  {\ttfamily mha\_set(openmha, \textquotesingle{}cmd\textquotesingle{},
                      \textquotesingle{}start\textquotesingle{} );}
\item \textbf{JACK Control}: Connect the "capture" and "playback" channels
  of the sound card to the MHA "in" and "out" channels.
  Connect a microphone to the soundcard.
\item 
\item $\rightarrow$ {\ttfamily mha\_get(openmha,\textquotesingle{}\textquotesingle{},\textquotesingle{}'mha.overlapadd.mhachain.acmon')}
\item $\rightarrow$ {\ttfamily mha\_get(openmha,\textquotesingle{}\textquotesingle{},\textquotesingle{}'mha.overlapadd.mhachain.acmon.coherence\_rcoh')}

\textbf{Gain\_live.cfg:}

\begin{Verbatim}[frame=single,numbers=left,commandchars=\\\{\}]
#The number of channels we want to process

\textcolor{orange}{\textbf{nchannels_in = 2}}
#Number of frames to be processed in each block.
\textcolor{orange}{\textbf{fragsize = 64}}
#Sampling rate. Has to be the same as the input signal of JACK
\textcolor{orange}{\textbf{srate = 44100}}
#Again, we want to use the plugin "mhachain"
\textcolor{orange}{\textbf{mhalib = transducers}}
Here we will only use one plugin "gain"
\textcolor{orange}{\textbf{mha.algos=[ gain ]}}
#Set max and min gain factors in dB
\textcolor{orange}{\textbf{mha.gain.min=-20}}
\textcolor{orange}{\textbf{mha.gain.max=20}}
#two gain factors (left and right)
\textcolor{orange}{\textbf{mha.gain.gains=[ -10 10 ]}}
#In this example, we load the IO library that connects
#the MHA to the Jack audio server.
\textcolor{orange}{\textbf{iolib = MHAIOJackdb}}



iolib = MHAIOJackdb

mha.plugin_name = overlapadd

mha.calib_in.peaklevel = [90 90]
mha.calib_out.peaklevel = [90 90]

mha.overlapadd.fftlen = 256
mha.overlapadd.wnd.len = 128








\end{Verbatim} 

\textbf{Matlab Script:}

\begin{Verbatim}[frame=single,numbers=left,commandchars=\\\{\}]
#The number of channels we want to process
\textcolor{orange}{\textbf{nchannels_in = 2}}
#Number of frames to be processed in each block.
\textcolor{orange}{\textbf{fragsize = 64}}
#Sampling rate. Has to be the same as the input signal of JACK
\textcolor{orange}{\textbf{srate = 44100}}
#Again, we want to use the plugin "mhachain"
\textcolor{orange}{\textbf{mhalib = mhachain}}
Here we will only use one plugin "gain"
\textcolor{orange}{\textbf{mha.algos=[ gain ]}}
#Set max and min gain factors in dB
\textcolor{orange}{\textbf{mha.gain.min=-20}}
\textcolor{orange}{\textbf{mha.gain.max=20}}
#two gain factors (left and right)
\textcolor{orange}{\textbf{mha.gain.gains=[ -10 10 ]}}
#In this example, we load the IO library that connects
#the MHA to the Jack audio server.
\textcolor{orange}{\textbf{iolib = MHAIOJackdb}}
\end{Verbatim} 















\end{enumerate}
%mha_get(openmha,'mha.overlapadd.mhachain.acmon')
\newpage

\end{document}

%%% Local Variables: 
%%% mode: latex
%%% TeX-master: "openMHA_application_manual"
%%% indent-tabs-mode: nil
%%% coding: utf-8-unix
%%% End:
