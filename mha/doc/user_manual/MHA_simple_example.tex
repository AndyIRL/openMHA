\section{A simple example configuration and how to start it}%
\label{sec:scenarios}%
\index{example configuration}%
\index{configuration!example}%
\index{configuration file}%

In this section, an example is shown on how to configure and start the
\mha{}. First, a simple algorithm is designed, which implements a 
linear hearing aid (equaliser). Then examples are given on
how to start this algorithm in different situations (\Windows{}
framework, \mhad{}, \Matlab{} processing).

\subsection{A linear hearing aid algorithm}%
\label{sec:linha}%

In this example, a \mha{} with a single chain will be configured. The
input signal will be equalised in five frequency bands, using the \mha{}
plugins \verb!fftfilterbank! and
\verb!gain!. The sound level at input and
output will be measured. Only spectral processing is required, thus
the plugin \verb!overlapadd! will be
used by the framework. The corresponding configuration file for this
plugin is created (you will find the final version \verb!mhaex1.cfg!
in the \verb!cfg!  directory of your \mha{} distribution).

\mha{} configuration files usually start with an identification line:
\begin{verbatim}
#[MHAVersion 4.4]
\end{verbatim}
Please note that since version 4.4, the first line in configuration
files is not treated as an identification line, thus a comment
character is required before the version number.

In the next lines, we configure the parameters of the overlap-add
method. A fragment size of 64 samples is assumed. Zeros will be padded
at both ends to avoid circular aliasing.

Within the overlapadd framework, we process a sequence of plugins
by setting the \verb!mhachain! plugin.
\begin{verbatim}
fftlen = 192
wnd.len = 128
plugin_name = mhachain
\end{verbatim}
A single chain will be processed, with a level meter at the input, a
frequency grouping into bands, a gain plugin, a filterbank re-synthesis
and a level meter at the output:
\begin{verbatim}
mhachain.algos = [ ...
  rmslevel:Lin ...
  fftfilterbank gain combinechannels ...
  rmslevel:Lout ...
]
\end{verbatim}
In the next step, the filter bank will be configured with five
overlapping octave bands. The filter shape will be a hanning shape on
a linear amplitude scale and a Bark-frequency scale.
\begin{verbatim}
mhachain.fftfilterbank.fscale = bark
mhachain.fftfilterbank.ovltype = hanning
mhachain.fftfilterbank.plateau = 0
mhachain.fftfilterbank.ftype = center
mhachain.fftfilterbank.f = [250 500 1000 2000 4000]
\end{verbatim}
Each filterbank band is handled as a separate channel. The next plugin
can adjust the gain of each channel in a limited range, here between
-16 and 32 dB. The gain plugin is followed by the filterbank
resynthesis.
\begin{verbatim}
mhachain.gain.min = -16
mhachain.gain.max = 32
mhachain.gain.gains = [-1 5 10 15 10 2 7 9 12 16]
mhachain.combinechannels.outchannels = 2
\end{verbatim}

We will assume, that this algorithm configuration was stored in the
file \verb!mhaex1.cfg!. Now we created a complete MHA hearing aid
algorithm configuration file. In the next step, we want to start a MHA
framework with this configuration.

\subsection{Using the \mhad{} and \Matlab{}}

The \mhad{} configuration interface can be easily accessed from the
\Matlab{} prompt through a generic graphical user interface,
'mhagui\_generic' (see also \secpageref{sec:mhagui_generic}).
This interface works on \Linux{} and \Windows{}. To start the \mhad{} on \Linux{}, type
\begin{verbatim}
MHA_LIBRARY_PATH=./
LD_LIBRARY_PATH=./
mha
\end{verbatim}
in the MHA 'bin' directory; on \Windows{} just double-click the file
'mha.exe'. Then start \Matlab{}, and add the MHA matlab directory to
your \Matlab{} search path using 'addpath'. Finally, before opening the user interface, add \verb!mhactl_java.jar! into your \Matlab{} classpath. To open the user interface, type
\begin{verbatim}
mhagui_generic
\end{verbatim}
at the \Matlab{} prompt.


%%%nowindows%% \subsection{Starting the \Windows{} framework 1: test executable}
%%%nowindows%% \label{sec:start-win-framewrk}
%%%nowindows%% 
%%%nowindows%% %\susubsection
%%%nowindows%% The following example uses the double buffering plugin 'db'. It is
%%%nowindows%% highly recommended to use double buffering on \Windows{} operating
%%%nowindows%% systems. However if you want to disable double buffering please refer
%%%nowindows%% to sections \ref{sec:framwork-win-exe}.
%%%nowindows%% 
%%%nowindows%% It is recommended to copy the shared fft libraries {\tt sfftw.dll} and
%%%nowindows%% {\tt srfftw.dll} from the bin/windows directory to a windows search
%%%nowindows%% path, e.g.\ \verb!C:\WINDOWS\SYSTEM!. Otherwise these libraries will
%%%nowindows%% not be found by the MHA if you use your own configurations and/or
%%%nowindows%% paths.
%%%nowindows%% 
%%%nowindows%% The \Windows{} test executable is configured with the file
%%%nowindows%% HearcomMHA.ini located in the same directory as the executable. This
%%%nowindows%% file is described in section \ref{sec:framwork-win-exe}, page
%%%nowindows%% \pageref{sec:framwork-win-exe}, here only some configuration entries
%%%nowindows%% that are specific for this example are described.
%%%nowindows%% 
%%%nowindows%% Here only the actual path is set as MHALibraryPath because all plugins
%%%nowindows%% are located in the same directory as HearComMHA.exe itself:
%%%nowindows%% \begin{verbatim}
%%%nowindows%% MHALibraryPath=;
%%%nowindows%% \end{verbatim}
%%%nowindows%% Double buffering is used, with the configuration file
%%%nowindows%% \verb!fwex3.cfg! (see below):
%%%nowindows%% \begin{verbatim}
%%%nowindows%% LibName=db
%%%nowindows%% ConfigFile=..\\..\\cfg\\fwex3.cfg
%%%nowindows%% \end{verbatim}
%%%nowindows%% Reasonable buffer size and buffer number. Not suited for low latency
%%%nowindows%% but 'safe':
%%%nowindows%% \begin{verbatim}
%%%nowindows%% BufferSize=2048
%%%nowindows%% NumBuffers=20
%%%nowindows%% \end{verbatim}
%%%nowindows%% Set overlap for FFT to 0.5. This will adjust the fragsize variable of
%%%nowindows%% the db plugin automatically (see sections \ref{sec:framwork-win-exe},
%%%nowindows%% \ref{sec:db} and \ref{sec:wave2spec}).
%%%nowindows%% \begin{verbatim}
%%%nowindows%% OLAFeed=0.5
%%%nowindows%% \end{verbatim}
%%%nowindows%% The example configuration is optimized for 48 kHz sampling
%%%nowindows%% frequency. ATTENTION: if you use the test executable to play files, be
%%%nowindows%% sure to use 48 kHz wave files for this example!
%%%nowindows%% \begin{verbatim}
%%%nowindows%% WaveInSampleRate=48000
%%%nowindows%% \end{verbatim}
%%%nowindows%% 
%%%nowindows%% On startup of the application the framework will load the plugin 'db'
%%%nowindows%% with its configuration file \verb!..\..\cfg\fwex3.cfg!. This contains
%%%nowindows%% just three lines (plus some comments on fragsize):
%%%nowindows%% \begin{verbatim}
%%%nowindows%% #[MHAVersion 4.4]
%%%nowindows%% plugname = concurrentchains
%%%nowindows%% plug?read:..\\..\\cfg\\mhaex1.cfg
%%%nowindows%% \end{verbatim}
%%%nowindows%% This causes the db plugin to load the MHA Kernel 'concurrentchains' and read the example
%%%nowindows%% configuration file \verb!..\\..\\cfg\\mhaex1.cfg! to the parser named
%%%nowindows%% 'plug' within the db plugin (see section \ref{sec:db}). Please note
%%%nowindows%% that the path to this configuration file has to be specified relative
%%%nowindows%% to the path where you started the application!
%%%nowindows%% 
%%%nowindows%% After the startup you may load files and play them through the MHA or
%%%nowindows%% perform direct I/O. For a description of the GUI please refer to
%%%nowindows%% section \ref{sec:framwork-win-exe} on page \pageref{sec:framwork-win-exe}.

%%\subsection{Starting SoundMex2 for \Matlab{} with the \mhalib{}}
%%\index{Matlab}%
%%\index{SoundMex}%
%%
%%To start the same example with the SoundMex2-MHA-Plugin for \Matlab{}
%%please run \Matlab{} and change the directory to
%%\verb!matlab\soundmex_mha!  within the distributions install directory
%%and run the example 'mha.m'. The framework (i.e. the plugin) is
%%configured with the file MHA.INI in the section [1]. This section
%%contains the same entries as the test executable configuration. Only
%%the \verb!BufferSize! and \verb!NumBuffers! entries are missing. They
%%have to be adjusted using the SoundMex2 command 'setbuffer' (please refer
%%to the SoundMex2 documentation for details).
%%
%%The only difference to the test executables configuration is the value
%%of the path entry:
%%\begin{verbatim}
%%MHALibraryPath=..\\..\\bin\\windows\\
%%\end{verbatim}
%%It points to the directory where the \Windows{} MHA plugins are
%%located.
%%
%%For a detailed description of the MHA-SoundMex2-Plugin please refer to
%%the separate manual "SoundMex2-MHA-Plugin.pdf".
%%
%%For a detailed description of SoundMex2 and the SoundMex2-DSP-Pipe
%%please refer to the separate manuals available from H\"orTech gGmbH.

\subsection{Starting the \mhad{} with JACK}%
\label{sec:example_jack}%
\index{JACK}%
\index{framework configuration}%
\index{configuration!framework}%

The configuration file for the \mhad{} has the same structure
as that of the algorithm. Again, the first line is an identification
string. Then we set the number of input channels, fragment size and
sampling rate of the framework. When using JACK, then fragment size
and sampling rate have to match those values used by the JACK server.
Here, we assume that the JACK server runs with a fragment size
of 64 samples, so that the overlap in the overlap-add method is
50\%. Later we will show how to use a double buffer for those
situations, where it is not possible to start JACK with the desired
fragment size.
\begin{verbatim}
#[MHAVersion 4.4]
nchannels_in = 2
fragsize = 64
srate = 48000
\end{verbatim}
%% In the next step, we have to calibrate the framework. For a linear
%% algorithm like an equalizer, the absolute level at the microphone or
%% receiver is not relevant. Therefore, it is sufficient to use the same
%% input and output calibration value (\verb!peaklevel_in!,
%% \verb!peaklevel_out!) for now. If knowledge of absolute values is
%% required, we have to measure the physical sound pressure level at the
%% microphone, and adjust the \verb!peaklevel_in!, until the internal
%% level meter shows that level. Similarely, the \verb!peaklevel_out! can
%% be adjusted. How to access the internal level meter will be described
%% below. Here comes the calibration section of the framework
%% configuration file:
%% \begin{verbatim}
%% peaklevel_in = [100 100]
%% peaklevel_out = [100 100]
%% \end{verbatim}
Now, we configure the libraries to use.
%
It is assumed, that the current working directory is "examples" from
the MHA distribution, and the binaries can be found in \verb!../bin/!,
i.e.\ the environment variables MHA\_LIBRARY\_PATH and LD\_LIBRARY\_PATH are configured
properly.
%
Please note that these environment variables have to be
terminated with a slash.
%
No extension is required in the configuration of plugin names.
\begin{verbatim}
iolib = MHAIOJack
mhalib = overlapadd
\end{verbatim}
Finally, we can connect the JACK client to the hardware input and
output ports and load the algorithm configuration file:
\begin{verbatim}
io.con_in = [alsa_pcm:capture_1 alsa_pcm:capture_2]
io.con_out = [alsa_pcm:playback_1 alsa_pcm:playback_2]
mha?read:mhaex1.cfg
\end{verbatim}
Please replace the port names by the ports you want to connect
to.

This framework configuration will be stored in the file
\verb!fwex1.cfg!. Before we can start the framework, we have to start
the license server and insert a valid license (dongle). Please install
the package \verb!license/HDD_Linux_USB_daemon.tar.gz! and start the
license server as described in that package.

The path to the MHA binaries should be exported to the \mhad{}.
Now, we start the \mhad{}:
%
\begin{verbatim}
export MHA_LIBRARY_PATH=../bin/
export LD_LIBRARY_PATH=../bin/
../bin/mha
\end{verbatim}
%
To start the \mhad{} from \Matlab{} on Linux, one can call
\begin{verbatim}
[errcode, pid] = system('mha & echo $!')
\end{verbatim}
(assuming, that the directory containing the MHA binaries is included
in the system path and in the MHA\_LIBRARY\_PATH environment
variable).
%
The variable \verb!pid! than contains the process id of the \mhad{} process as text.
%
The MHA configuration and control
input is accepted over a network connection. 
%
When no command line parameters are given, the default port number
33337 and the loopback network interface 127.0.0.1 is used, i.e., only
connections from the local host are accepted (see section
\ref{sec:linuxmhaserver} on page \pageref{sec:linuxmhaserver} for
details).
%
To enter MHA commands, please start a network client, e.g.\ telnet, to
open a MHA console:
\begin{verbatim}telnet localhost 33337\end{verbatim}

To read the framework configuration file, please type
\begin{verbatim}
?read:fwex1.cfg
\end{verbatim}
followed by the return key. If everything went well, the MHA will print
\verb!(MHA:success)!. In case of an error message \verb!(MHA:failure)!, MHA will also indicate the line containing the error. You need to correct this error using an editor you prefer and  reload it. If you receive anerror message, \verb!(mha_parser) (mha_parser) The variable is locked!, you need to close the \mhad{} and relaunch it. If the JACK server was not started yet, this is the
right moment to start your JACK server with the correct settings,
e.g.:
\begin{verbatim}
jackd -d alsa -r 48000 -p 64
\end{verbatim}
Please make sure that the fragment size and sample rate of the JACK
sound server matches the MHA fragment size (see below if it
doesn't). After having successfully started the JACK server, the MHA
can be started by typing
\begin{verbatim}
cmd = start
\end{verbatim}
at the MHA console. The processing can be stopped at any time by
typing \verb!cmd = stop!.

Now, we want to access the variables of the algorithm. The easiest way
is to type \verb!?!, followed by the return key, in the console. This
will show the complete MHA configuration, including all framework
variables and plugin configuration. Usually, this
produces so much output, that the console has to be scrolled to see
the complete information. If only a subset or a single variable is of
interest, the prefix of that subset or variable can be put before the
\verb!?!, e.g.\ all variables of the processing chain can be reached
by typing \verb!mha.mhachain?!, the peak and RMS levels of the input level
meter by typing \verb!mha.mhachain.Lin?!. All monitor variable contents can
be stored into the file "example.mon" by typing
\verb!?savemons:example.mon!. The \mhad{} will be closed
by typing \verb!cmd = quit!.

\paragraph{Adjusting the fragment size}%
\index{fragment size}%
\index{double buffering}%

If the required fragment size is not supported by the audio hardware,
double buffering can be used in the MHA frameworks. We assume now,
that the JACK server was started with a fragment size of 256 samples at
a sampling rate of 48 kHz, e.g.\ \verb!jackd -d alsa -r 48000 -p 256!. A
fragment size of 64 samples in the algorithm can be reached by inserting
a double buffer plugin between the framework and the algorithm. This is
done by replacing the MHA library \verb!overlapadd! by the double
buffer plugin \verb!db!, which will load \verb!overlapadd! as a
client. In the framework configuration in \verb!fwex1.cfg!, the line
\verb!mhalib = overlapadd! will be replaced by
\begin{verbatim}
mhalib = db
mha.plugin\_name = overlapadd
mha.fragsize = 64
\end{verbatim}
The fragment size of the framework will be set to that of the JACK
server, so please replace \verb!fragsize = 64! on the top level by
\verb!fragsize = 256!. Now, the configuration of the algorithm can be
found one layer below the former level, so the read command will be
modified to
\begin{verbatim}
mha.overlapadd?read:mhaex1.cfg
\end{verbatim}
Similarly, the variable contents of the processing chain can now be
queried by typing \verb!mha.overlapadd.mhachain?!.

\subsection{Start the MHA for real-time processing with \Matlab{} from Linux}%
\index{Matlab}%

The configuration\index{framework
configuration}\index{configuration!framework} of the \mhad{} can be
read at startup time by adding a MHA configuration language command:
%
\begin{verbatim}
../bin/mha '?read:fwex1.cfg' 'cmd=start'
\end{verbatim}
%
We assume, that your \Matlab{} process is running on the same host and
as the same user as the MHA, and that the \mhad{} runs with the
default port number 33337.
%
The MHA \Matlab{} tools directory has to be in the \Matlab{} path.
%
Now, communication with the configuration interface of the MHA is
possible through MHA \Matlab{} functions (see \secpageref{sec:matlab}
for a detailed documentation):
%
First, create a MHA connection handle for the \Matlab{} tools by
typing
\begin{verbatim}
h = struct( 'port', 33337, 'host', 'localhost' );
\end{verbatim}
at the \Matlab{} prompt. Then this handle can be used to connect to
the MHA:
%
\begin{verbatim}
result = mha_get( h, '' );
\end{verbatim}
%
The complete MHA configuration hierarchy is converted into a \Matlab{}
'struct' variable.
%
\Matlab{} vectors can be sent to the JACK server (and that way to the
MHA) with the use of the tool \verb!jackiomex! (see
\secpageref{sec:jackiomex}).
%
The next four lines plot the frequency response of our linear hearing
aid ('zeropad' and 'realfft' are functions from the MHA \Matlab{}
toolbox):
\begin{verbatim}
x = zeropad( 1, 48000 );
y = jackiomex( x, {'MHA:out_1'}, {'MHA:in_1'} );
plot( 20*log10( abs( realfft( y ) ) ) );
set( gca, 'XScale', 'log', 'XLim', [40 20000], 'YLim', [-5 16] );
\end{verbatim}
When the MHA processing is not needed any more, the MHA can be shut
down by calling
%
\begin{verbatim}
mha_set( h, 'cmd', 'quit' );
\end{verbatim}


%%% Local Variables:
%%% mode: latex
%%% TeX-master: "MHA_manual"
%%% End:

% LocalWords:  mhactl MHA Matlab jackiomex plugin MHAIOJack iolib mhalib alsa
% LocalWords:  overlapadd pcm io mha mhaex cfg fwex errcode pid savemons mon
% LocalWords:  cmd jackd concurrentchains plugname fragsize struct localhost
% LocalWords:  eval zeropad XLim realfft gca XScale YLim
