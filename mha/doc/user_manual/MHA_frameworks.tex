%\section{The MHA Frameworks}

\section{The \mhad{}}\label{sec:linuxmhaserver}\label{sec:frameworks}

The \mhad{} ('mha' on \Linux{}) provides a control interface for 
the configuration and connects to the audio abstraction layer (MHA IO).
%
The text based user interface is available through a TCP network
socket.
%
External network clients, e.g.\ telnet, Netcat or the \Matlab{} control
interface function 'mhactl' (see \secpageref{sec:mhactl}) can be used
to access this interface.
%
Multiple IO modules are available in the audio abstraction layer,
which hide the platform dependency (see
\secpageref{sec:audioabstraction}).
%

The \mhad{} and all of its plugins can be configured with the \mha{}
configuration language (see \secpageref{sec:script} and \secpageref{sec:scenarios}).
%


\subsection{Invocation of 'mha'}

If the \mhad{} is invoked without any command line arguments, it
starts a network service on TCP port 33337, loopback network interface,
accepting connections from the local host, expecting configuration language
commands.
%
The behaviour of the server can be controlled through a set of command
line options:
%
\begin{description}
\item\verb!--quiet | -q!\\Suppress the output, do not show any
greeting text or error messages.
\item\verb!--port=portno | -s portno!\\Set the port number to which
  the \mhad{} should bind (default: 33337). 
  If port number is 0, 
  then the operating system chooses a free port for the mha to bind to.
\item\verb!--announce=port | -a port!\\ If given, then the \mha{} connects to 
  this TCP port on the localhost after it has established its own TCP server 
  socket, and announces its process ID and the TCP server port in use, and
  closes the connection again. 
\item\verb!--interface=if | -i if!\\Set the network interface to which
the \mhad{} should bind (default: 127.0.0.1).
\item\verb!--daemon | -d!\\Start the \mhad{} in daemon
  mode. This means that after a \mha{} server was closed (via the \mha{}
  command 'cmd=quit'), the \mhad{} will wait for a new
  connections. In daemon mode the \mhad{} can be stopped by killing
  the daemon process or by pressing {\tt Ctrl-C} at the console.
\item\verb!--ok-ack=str | -o str!\\Set the acknowledgement string for
  accepted \mha{} command lines (default value is '(MHA:success)').
\item\verb!--fail-ack=str | -f str!\\Set the acknowledgement string
  for rejected \mha{} command lines (default value is
  '(MHA:failure)').
\item\verb!--log=logfile!\\Set the log file to 'logfile'
  (default: /dev/null).
\item\verb!--help | -h!\\Print an overview about the command line
  arguments.
\item\verb!--lockstr=str | -l str!\\Create a file with name 'portno'
and write the text 'str' into that file. The file is removed after the
\mha{} session is closed.
\item\verb!--license!\\Print the license agreement.
\end{description}

Additional command line arguments which are not recognised as options
will be interpreted as \mha{} configuration language commands and sent to the
\mhad{} after allocation, before accepting other input.
%
In daemon mode, these \mha{} configuration language commands are interpreted at the 
start of each session.

\verb!mha --daemon ?read:defaults.cfg! will read a default
configuration file for each session.
%
Clients for the \mhad{} are the \Matlab{} tool 'mhactl' and any
telnet client (not part of the distribution).

The \mhad{} searches for \mha{} plugins in the system library paths, or in
the directories given in the environment variable {\tt
MHA\_LIBRARY\_PATH}.
%
Multiple paths can be separated by a semicolon.

\paragraph{Warning}

The \mhad{} accepts connections from any host that can reach
the configured network interface.
%
Sender authentication and transport encryption is not implemented.
%
We therefore strongly recommend to use the \mhad{} only in a
physically separated network or behind a firewall.
%
We explicitly do not take any liability in case of abuse of patient
data transmitted to the \mhad{} or any other interference.

Please do not modify the acknowledgement strings if a communication
with the \Matlab{} tool 'mhactl' is required.

\subsection{Configuration variables of the \mhad{}}

In the following list the configuration variables of the \mhad{} are
described. These variables are accessible through the parser interface
(e.g.\ console input, TCP).
%
A configuration file with these settings can be read by sending a {\tt
?read:filename.cfg} command to the configuration interface. See also
\secpageref{sec:script} for details.

\begin{description}
  \mhavardesc{nchannels\_in}{Number of input audio channels.}
  
  \mhavardesc{fragsize}{The fragment size in samples per audio
    channel. If 'MHAIOJack' is used, this has to match the JACK
    fragment size (see \secpageref{sec:scenarios} for an example).}
  
  \mhavardesc{srate}{Sampling rate in Hz. Please note that JACK allows only a
    fixed sampling rate given at the invocation of 'jackd'.}
  
  \mhavardesc{mhalib}{The MHA processing library name (e.g.\ 'transducers', 'mhachain'  or 'db').}
  
  \mhavardesc{iolib}{The IO plugin library name (e.g.\ 'MHAIOJack' or
  'MHAIOFile'), see \secpageref{sec:audioabstraction}.}
  
  \mhavardesc{cmd}{This variable controls the operation state of the
    \mhad{}. The valid states (nop, prepare, start, stop, release,
    quit) of the \mhad{} are described in \secpageref{sec:fwstates}.}

  \mhavardesc{mha}{This subparser contains the configuration of the
    processing library.}
  
\mhavardesc{io}{This subparser contains the configuration of the IO
    library.}

\mhavardesc{sleep}{This special command waits on the normal execution of commands
   while \mha{} continues processing audio. The number of seconds waited is given by the right-hand side
   e.g. {\tt sleep = 5} waits 5 seconds.}

\end{description}

\subsection{States of the \mhad{}}%
\label{sec:fwstates}\index{states}%

The states of the \mhad{} are controlled by setting the {\tt cmd}
variable. After configuring all modules of the \mha{} (Framework and
Plugins), the configuration can be prepared to be ready for signal
processing by setting {\tt cmd=prepare}. This will also validate the
configuration; on error the prepare command will be rejected. Setting
{\tt cmd=start} tells the IO plugin to start the signal processing,
and accordingly setting {\tt cmd=stop} will cause the IO plugin to
stop processing. Invoking {\tt cmd=release} brings the IO plugin into
an unlocked state. The session can be closed with {\tt cmd=quit}. See
also \figref{MHA-fw-states}.

\MHAfigure[][0.7\linewidth]{States of the \mhad{}}{MHA-fw-states}

\subsection{Audio abstraction layer}%
\label{sec:audioabstraction}

The audio abstraction layer connects the audio backbone, i.e.,
Jack (see \secpageref{sec:jack}) or audio files, with the \mhad{}.
%
This layer consists of two modules: 'MHAIOJack' for low delay real
time processing with the JACK audio server (see \secpageref{sec:jack})
on \Linux{} and 'MHAIOFile' for file to file processing (both platforms).

\paragraph{The 'MHAIOJack' audio IO module}%
\index{MHAIOJack}%
\index{Jack Audio Connection Kit}%

The module 'MHAIOJack' provides communication with the JACK audio
server (see \secpageref{sec:jack}).
%
When the \mhad{} is prepared for processing, this module connects to a
running JACK server and validates its parameters. The input and output
ports of the MHA can be connected to any other JACK ports through the
\mha{} configuration (see below) or externally. Please note, that
MHAIOJack currently supports only fixed sample rates and fragment
sizes. Changing the fragment size of JACK while processing will stop
the \mha{} processing thread. If sample rates are required which are not
provided by JACK, the {\tt downsample} and {\tt upsample} plugins can
be used. 

Variables of the 'MHAIOJack' module:

\begin{description}
\mhavardesc{name}{Name of the JACK client. This variable only needs to
  be modified if multiple instances of MHAs should run simultaneously.}

\mhavardesc{con\_in}{Connection list for input \mha{} ports with one
  entry for each port, e.g.\ {\tt con\_in = [alsa\_pcm:capture\_1
    alsa\_pcm:capture\_2]}. The ports are reconnected at any time the
  variable is accessed. Ports can be disconnected by using a colon as
  a port name. To achieve multiple connections to one \mha{} port, please
  use external connection tools, e.g. 'qjackctl' or 'jack\_connect'.}

\mhavardesc{con\_out}{Connection list for output \mha{} ports with one
  entry for each port, e.g.\
  {\tt con\_out = [alsa\_pcm:playback\_1 alsa\_pcm:playback\_2]}.}

\mhavardesc{names\_in}{Labels of \mha{} input ports (empty for auto-generated labels).}

\mhavardesc{names\_out}{Labels of \mha{} output ports (empty for auto-generated labels).}

\end{description}

In the node \verb!ports!, monitor variables filled with available
hardware and software ports of Jack can be found.

\MHAfigure[][\linewidth]{Typical session using the \mhad{} and Jack}{linuxfw_screenshot}

\paragraph{The 'MHAIOFile' audio IO module}%
\index{MHAIOFile}%
\index{file processing}%
\index{audio file}%
\index{file!audio}%
\label{sec:MHAIOFile}%

The module 'MHAIOFile' provides file to file processing with the
\mha{}. Input and output file name can be configured. After the \mhad{}
is started (\verb!cmd=start!), the whole input file will be
processed and the processed data will be written to the output
file. The start command will wait until the processing is finished.
The files are opened when preparing the \mhad{} and closed when
releasing the \mhad{}. The file and data format of the output file
is inherited from the input file, e.g. if the input file is a 32 bit
WAVE file, also the output file will be. The plugin supports most
commonly used file formats.

The variables of 'MHAIOFile' are:

\begin{description}
\mhavardesc{in}{Input file name.}
\mhavardesc{out}{Output file name.}
\mhavardesc{output\_sample\_format}{Output sample format, or 'input' to copy format specification from input file.}
\mhavardesc{startsample}{First sample to be processed.}
\mhavardesc{length}{Number of samples to be processed by one start command, or zero for all.}
\mhavardesc{strict\_channel\_match}{Require same channel count in \mha{} and input sound file. If yes, an error message is created if the channel count doesn't match, otherwise additional channels are ignored and missing channels are filled with zeros.}
\mhavardesc{strict\_srate\_match}{Require same sample rate in \mha{} and sound file. If yes then an error is reported if the sample rate does not match, otherwise the sample rate of the sound file is ignored (no re-sampling).}
\end{description}

%%% Local Variables: 
%%% mode: latex
%%% TeX-master: "MHA_manual"
%%% End: 
