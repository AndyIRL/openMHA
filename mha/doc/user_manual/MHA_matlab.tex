\section{\Octave{}/ \Matlab{} tools}%
\label{sec:matlab}%

In this package release \mha{} related tools for usage with \Octave{} and \Matlab{} are included.
%
No support
is granted for these modules, nor give we any warranty for usage of
these tools.

The \mhad{} can be controlled through a simple \Octave{}/ \Matlab{}
interface (mhactl). This tool opens a TCP connection to a \mhad{} and
communicates with the framework configuration interface.
%
For data exchange with the \mha{}, an \Octave{}/ \Matlab{} client to the JACK low latency sound server (see section \ref{sec:jack} on page
\pageref{sec:jack}) is provided within this release.
%
This interface gives direct access to the low latency real-time
processing system from \Octave{} and \Matlab{} without requiring special toolboxes.

Algorithm communication variables can be exported to \Matlab{}-format files
using the 'acsave' algorithm.

\subsection{"mhactl\_wrapper" - \mha{} control interface for \Octave{} and \Matlab{}}
\label{sec:mhactl}

The  \Octave{}/ \Matlab{} function \verb!mhactl_wrapper! communicates with the \mhad{}
through a TCP network connection.
%
For correct operation, the \mhad{} has to be started with the
default acknowledge/prompt strings.
%
It is not required that the MHA process runs as the same user or on
the same machine as \Octave{} or Matlab{}.

The function 'mhactl\_wrapper' accepts two arguments, the \mha{} handle 
(struct with the correct TCP port and host), and the \mha{} query to be processed:
\verb!result = mhactl( mha_handle, query )!
%
The 'mhactl\_wrapper' function opens a network connection to the \mhad{},
and sends the command string to the MHA and waits for an acknowledge
prompt.
%
On success, the MHA response (without the acknowledge prompt) is
returned, otherwise an error is reported.

\subsection{Wrapper functions for "mhactl\_wrapper"}
\label{sec:mhactl_wrapper}

While 'mhactl\_wrapper' provides direct access to the \mha{} control interface,
some wrapper functions are implemented which utilize 'mhactl\_wrapper' to
convert \mha{} control commands into \Octave{}/ \Matlab{} values and back.

\subsubsection{"mha\_get" - read contents of a \mha configuration}

The function 'mha\_get' reads the contents of an \mha{} configuration
entry and returns them in a \Octave{}/ \Matlab{} type, i.e., a type dependent
conversion from the \mha{} string representation is performed. The
command syntax is
\begin{verbatim}
[answer, info] = mha_get(handle, field, perm ).
\end{verbatim}
The \mha{} handle 'handle' is a structure containing the fields 'host'
and 'port' defining the host name and port number of the \mhad{}.
%
'field' is the name of the \mha{} configuration entry.
%
It can be either a variable or a parser node -- in the first case, the
content of the variable is returned in 'answer' and the help comment
of the variable is returned in 'info', if available.
%
If 'field' denotes a parser node, 'answer' will hold a \Octave{}/ \Matlab{}
structure, with each field holding the contents of an \mha{} variable or a
sub-parser.
%
In this situation, it is possible to restrict the query only to
entries with a specific permission, which can be given in 'perm'.
%
'perm' can be either a character string, or a cell array of string.
%
To receive the complete writable configuration of an \mhad{}, type
\verb!cfg = mha_get( handle, '', 'writable' )!
%

\subsubsection{"mha\_set" - set contents of \mha{} configuration entries}

\Octave{}/ \Matlab{} values can be assigned to \mha{} configuration entries via the
'mha\_set' function.
%
The syntax of this function is:
\verb!mha_set( handle, field, value )!
%
As in 'mha\_get', 'handle' is a structure containing the fields 'host'
and 'port' defining the host name and port number of the \mhad{}, and
'field' is the name of the \mha{} configuration entry.
%
The parameter 'value' is a \Matlab{} representation to be assigned to
the variable 'field'.
%
The \Octave{}/ \Matlab{} representation is converted to the correct \mha{} string
representation by first retrieving the type of the configuration entry
'field' through the control interface.
%
If the \Octave{}/ \Matlab{} value cannot be converted, an error is reported.
%
To setup a complete \mha{}, it is possible to assign a \Octave{}/ \Matlab{}
configuration structure 'cfg' to the \mha{} by typing
\verb!mha_set( handle, '', cfg )!

\subsection{"mhagui\_generic" - Generic graphical user interface}
\label{sec:mhagui_generic}

A generic graphical user interface (GUI) to the \mhad{} is available
via the function \verb!mhagui_generic! and the helper functions
\verb!mhagui_*.m!.
%
The syntax of the GUI function is:
\begin{verbatim}
h = mhagui_generic( handle, base )
\end{verbatim}
%
As before, 'handle' is a structure containing the fields 'host' and
'port' defining the host name and port number of the \mhad{}. The
default values are 'localhost' and 33337.
%
'base' is the name of the \mha{} parser node (default: '', i.e.\ root level).
%
A control panel is created in a \Octave{}/ \Matlab{} figure, and the figure
handle is returned.
%
A control element for each entry in the parser 'base' is created.
%
Numeric scalars are represented as sliders, keyword lists as select
boxes and boolean entries as toggle buttons.
%
For vectors of floating point values, a window with a slider array can
be opened.
%
Sub-parser can be opened as a new window, containing an own control
panel.
%
Other types can be edited in a text editing field.

If the \mha{} is running on the same host as the \Octave{}/ \Matlab{} control
interface, it is possible to read and save \mha{} configuration files by
clicking the 'read' or 'save' button. The read/save command operates
relative to the \mha{} parser level displayed in the control panel, i.e.,
the complete configuration should be read or saved from the root level
panel.

\MHAfigure[][.8\linewidth]{Generic graphical user interface of a
\mhad{}, created under \Octave{}/ with the function
'mhagui\_generic'.}{mhagui_generic}

%%% Local Variables: 
%%% mode: latex
%%% TeX-master: "openMHA_application_manual"
%%% End: 

% LocalWords:  mhactl jackiomex MHA mhagui mha cfg mex csInPorts csOutPorts
