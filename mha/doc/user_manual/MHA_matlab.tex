\section{\Matlab{} tools}%
\label{sec:matlab}%

Officially, \Matlab{} is not supported by the MHA.
%
However, in this
package several MHA related tools for \Matlab{} are included.
%
No support
is granted for these modules, nor give we any warranty for usage of
these tools.
%
The Linux Mex files have been developed and tested with \Matlab{} 7.0
(R14SP2), using GCC-3.3.
%
To solve a compiler incompatibility with \Matlab{} R14SP2, a
work-around had to be applied%
\footnote{%
%
This problem and a workaround is described in the support website of
Mathworks, see e.g.\
\url{http://www.mathworks.com/support/solutions/data/1-QBCS1.html?1-QBCS1}.
%
}.
%
Mathworks stated that this problem has been fixed with \Matlab{} R14SP3.


The \mhad{} can be controlled through a simple \Matlab{}
interface (mhactl). This tool opens a TCP connection to a \mhad{} and
communicates with the framework configuration interface.
%
For data exchange with the MHA, a \Matlab{} client to the JACK low
latency sound server (see section \ref{sec:jack} on page
\pageref{sec:jack}) is provided within this distribution.
%
This interface gives direct access to the low latency real time
processing system from \Matlab{} without requiring special toolboxes.

Algorithm communication variables can be exported to \Matlab{} files
using the 'acsave' algorithm.

\subsection{"mhactl" - MHA control interface for \Matlab{}}
\label{sec:mhactl}

The \Matlab{} mex-function \verb!mhactl! communicates with the \mhad{}
through a TCP network connection.
%
For correct operation, the \mhad{} has to be started with the
default acknowledge/prompt strings.
%
It is not required that the MHA process runs as the same user or on
the same machine as \Matlab{}.

The mex function 'mhactl' accepts three arguments, the command to be
processed, and the port number and host name of the \mhad{}:
\begin{verbatim}handle = mhactl( cmd, port, host )\end{verbatim}
If the port number and the host name are omitted, the default values
(port: 33337, host: 'localhost') will be used. The command argument
\verb!cmd! can contain exactly one MHA command string or an empty
string.
%
The 'mhactl' function opens a network connection to the \mhad{},
and sends the command string to the MHA and waits for an acknowledge
prompt.
%
On success, the MHA response (without the acknowledge prompt) is
returned, otherwise an error is reported.

\subsection{\Matlab{} wrapper functions for "mhactl"}
\label{sec:mhactl_wrapper}

While 'mhactl' provides direct access to the MHA control interface,
some wrapper functions are implemented which utilize 'mhactl' to
convert MHA control commands into \Matlab{} values and back.

\subsubsection{"mha\_get" - read contents of a MHA configuration}

The function 'mha\_get' reads the contents of a MHA configuration
entry and returns them in a \Matlab{} type, i.e., a type dependent
conversion from the MHA string representation is performed. The
command syntax is
\begin{verbatim}[answer, info] = mha_get(handle, field, perm )\end{verbatim}
The MHA handle 'handle' is a structure containing the fields 'host'
and 'port' defining the host name and port number of the \mhad{}.
%
'field' is the name of the MHA configuration entry.
%
It can be either a variable or a parser node -- in the first case, the
content of the variable is returned in 'answer' and the help comment
of the variable is returned in 'info', if available.
%
If 'field' denotes a parser node, 'answer' will hold a \Matlab{}
structure, with each field holding the contents of a MHA variable or a
sub-parser.
%
In this situation, it is possible to restrict the query only to
entries with a specific permission, which can be given in 'perm'.
%
'perm' can be either a character string, or a cell array of string.
%
To receive the complete writable configuration of a \mhad{}, type
\begin{verbatim}cfg = mha_get( handle, '', 'writable' )\end{verbatim}
%

\subsubsection{"mha\_set" - set contents of MHA configuration entries}

\Matlab{} values can be assigned to MHA configuration entries via the
'mha\_set' function.
%
The syntax of this function is:
\begin{verbatim}mha_set( handle, field, value )\end{verbatim}
%
As in 'mha\_get', 'handle' is a structure containing the fields 'host'
and 'port' defining the host name and port number of the \mhad{}, and
'field' is the name of the MHA configuration entry.
%
The parameter 'value' is a \Matlab{} representation to be assigned to
the variable 'field'.
%
The \Matlab{} representation is converted to the correct MHA string
representation by first retrieving the type of the configuration entry
'field' through the control interface.
%
If the \Matlab{} value cannot be converted, an error is reported.
%
To setup a complete MHA, it is possible to assign a \Matlab{}
configuration structure 'cfg' to the MHA by typing
\begin{verbatim}mha_set( handle, '', cfg )\end{verbatim}

\subsection{"mhagui\_generic" - Generic graphical user interface}
\label{sec:mhagui_generic}

A generic graphical user interface (GUI) to the \mhad{} is available
via the function \verb!mhagui_generic! and the helper functions
\verb!mhagui_*.m!.
%
The syntax of the GUI function is:
\begin{verbatim}
h = mhagui_generic( handle, base )
\end{verbatim}
%
As before, 'handle' is a structure containing the fields 'host' and
'port' defining the host name and port number of the \mhad{}. The
default values are 'localhost' and 33337.
%
'base' is the name of the MHA parser node (default: '', i.e.\ root level).
%
A control panel is created in a \Matlab{} figure, and the figure
handle is returned.
%
A control element for each entry in the parser 'base' is created.
%
Numeric scalars are represented as sliders, keyword lists as select
boxes and boolean entries as toggle buttons.
%
For vectors of floating point values, a window with a slider array can
be opened.
%
Sub-parser can be opened as a new window, containing an own control
panel.
%
Other types can be edited in a text editing field.

If the MHA is running on the same host as the \Matlab{} control
interface, it is possible to read and save MHA configuration files by
clicking the 'read' or 'save' button. The read/save command operates
relative to the MHA parser level displayed in the control panel, i.e.,
the complete configuration should be read or saved from the root level
panel.

\MHAfigure[][\linewidth]{Generic graphical user interface of a
\mhad{}, created by the \Matlab{} function
'mhagui\_generic'. See text for details.}{mhagui_generic}

\subsection{"jackiomex" - \Matlab{} interface to JACK}
\label{sec:jackiomex}

The \Matlab{} mex-function \verb!jackiomex! plays and records a
\Matlab{} buffer. Playback and capture are synchronized with sample
resolution. The buffer can have any number of channels; the output
buffer will have the same dimension as the input buffer. The JACK
ports to which the playback/capture channels should be connected are
given by the cell string arrays \verb!csInPorts! and
\verb!csOutPorts!. If an entry is empty or less entries are given than
channels in the buffer, then the port will not be connected
automatically (please use external connection tools if more complex
connections are required).

Three transport modes are supported. The first one immediately starts
the recording. The first recorded samples might be zero if the
connections are not established when recording the first frames, even
with automatic connections. Other transport modes make use of the JACK
transport system and therefore allow synchronization with external
JACK clients at sample resolution.

The resolution of the recorded data and the delay between input and
output depends on your hardware settings. Using a professional audio
device you can reach delays below 2 ms at 24 bit resolution. The input
and output buffers are internally stored as double precision floats
and are normalized to one.

\begin{verbatim}
[y, r] = jackiomex( x, csInPorts, csOutPorts [, JackTransport ] )
\end{verbatim}

\paragraph{Parameters:}
\begin{description}
\item{\tt x} input matrix (one column for each channel)
\item{\tt csInPorts} cell string array with input port names
\item{\tt csOutPorts} cell string array with output port names
\item{\tt JackTransport} optional transport mode flag:\\
0 = start immediately (default)\\
1 = play/record while JACK transport is rolling\\
2 = as 1, but transport is started and stopped by jackiomex
\end{description}
\paragraph{Return values:}
\begin{description}
\item{\tt y} output signal (same dimension as input signal)
\item{\tt r} JACK sample rate in Hz
\end{description}

\paragraph{Warning:}

Please note that starting JACK with real-time priority and
memory-locking might cause to kill your \Matlab{} process or freeze your
system, because it might not be possible to lock the complete \Matlab{}
memory. It is recommended not to use jackiomex when memory locking is
active.


%%% Local Variables: 
%%% mode: latex
%%% TeX-master: "MHA_manual"
%%% End: 

% LocalWords:  mhactl jackiomex MHA mhagui mha cfg mex csInPorts csOutPorts
