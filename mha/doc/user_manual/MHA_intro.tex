\section{Introduction}
 
The H�rTech Open Master Hearing Aid (\mha{}) is software development
platform for signal processing algorithms and a software solution of a
hearing aid prototype.
%
The Master Hearing Aid has a layered structure as shown in 
\figref{structure_openmha}.

\MHAfigure[][0.4\linewidth]{Layered structure of the master hearing aid}{structure_openmha}

\begin{itemize}

\item The \textbf{audio backbone} is external to the master hearing
aid.
It is an abstraction for a source and a sink for audio data. 
The \mha{} can use different audio backbones by selecting the corresponding
\mha{} IO library.
Usable audio backbones are: 
The file system\footnote{several sound file formats are supported through 
the use of the libsndfile library}, 
sound card drivers (Windows MME, Windows ASIO, Linux Alsa, Linux OSS), 
the Jack low latency sound server and router, 
and network audio streams. 

\item The {\bf \mha{} Framework}: selects the \mha{} IO library
and passes the fragmented signal to the \mha{} plugins. The
frameworks are platform dependant programs or libraries. Towards the
user, a control interface is provided.
 
\item An {\bf \mha{} Plugin} generally performs the
signal processing, but plugins for signal flow management also exist.
Each plugin forms one
processing block; algorithms can consist of one or more plugins. \mha{}
plugins communicate with the framework and with other plugins 
through a simple ANSI-C interface. Plugins can work on waveform data
(time domain) or STFT (frequency domain). 
Plugins can implement domain translations, 
e.g. FFT (fast fourier transform) and iFFT (inverse FFT) plugins exist.
\end{itemize}

Additionally, many tools of different kind are part of the \mha{}
software development platform: A toolbox library for easy development
of real-time capable algorithms, many \mha{} Plugins for signal analysis
and basic signal modification, and applications to access the
configuration space (e.g.\ GUI configurator, \Matlab{} access).

An example of the possible signal flow in the \mha{} is shown in
\figref{multilayer}. The multi-channel input signal, symbolised by a
set of microphones, is read by the framework and passed to the plugin
'concurrentchains'. There, the data is duplicated and passed to all
chains as a waveform stream, or if requested by a processing chain, as
short time Fourier transform (STFT) data. The user can select the chain to be
played back; switching is allowed at any time.  If the last algorithm
in the selected chain returns STFT, an inverse Fourier-transform is
applied to the data.

\medskip

The structure of this user manual is as follows: First (section
\ref{sec:variables}), the variable concept is described for
clarification of the nomenclature. Then, the \mha{} frameworks (section
\ref{sec:frameworks}), followed by some example scenarios and a
step-by-step description on how to start the \mha{} (section
\ref{sec:scenarios}). Afterwards, a description of some \Matlab{}
tools is given (section \ref{sec:matlab}). Finally, a section on
tuning a low latency signal processing environment is added (section
\ref{sec:system}). The Programmers Reference Handbook describes the
programming interface of the \mha{} and the toolbox. \mha{} plugins are
documented in a separate document.

\MHAfigure[][0.7\linewidth]{Schematic typical signal flow in
  the Master Hearing Aid}{multilayer}

%%% Local Variables: 
%%% mode: latex
%%% TeX-master: "MHA_manual"
%%% End: 
