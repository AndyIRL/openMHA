%%% This file is part of the Open HörTech Master Hearing Aid (openMHA)
%%% Copyright © 2007 2008 2010 2011 2012 2013 2016 2017 2018 HörTech gGmbH
%%%
%%% openMHA is free software: you can redistribute it and/or modify
%%% it under the terms of the GNU Affero General Public License as published by
%%% the Free Software Foundation, version 3 of the License.
%%%
%%% openMHA is distributed in the hope that it will be useful,
%%% but WITHOUT ANY WARRANTY; without even the implied warranty of
%%% MERCHANTABILITY or FITNESS FOR A PARTICULAR PURPOSE.  See the
%%% GNU Affero General Public License, version 3 for more details.
%%%
%%% You should have received a copy of the GNU Affero General Public License, 
%%% version 3 along with openMHA.  If not, see <http://www.gnu.org/licenses/>.

% Latex header for doxygen 1.8.11
\documentclass[11pt,a4paper,twoside]{article}

% Packages required by doxygen
\usepackage{fixltx2e}
\usepackage{calc}
\usepackage{../openMHAdoxygen}
\setlength{\headheight}{13.6pt}
\usepackage[export]{adjustbox} % also loads graphicx
\usepackage{graphicx}
\usepackage[utf8]{inputenc}
\usepackage{makeidx}
\usepackage{multicol}
\usepackage{multirow}
\PassOptionsToPackage{warn}{textcomp}
\usepackage{textcomp}
\usepackage[nointegrals]{wasysym}
\usepackage[table]{xcolor}

% Font selection
\usepackage[T1]{fontenc}
\usepackage{helvet}
\usepackage{courier}
\usepackage{amssymb}
\usepackage{sectsty}
\renewcommand{\familydefault}{\sfdefault}
\allsectionsfont{%
  \fontseries{bc}\selectfont%
  \color{darkgray}%
}
\renewcommand{\DoxyLabelFont}{%
  \fontseries{bc}\selectfont%
  \color{darkgray}%
}
\newcommand{\+}{\discretionary{\mbox{\scriptsize$\hookleftarrow$}}{}{}}

% Page & text layout
\usepackage{geometry}
\geometry{%
  a4paper,%
  top=2.5cm,%
  bottom=2.5cm,%
  left=2.5cm,%
  right=2.5cm%
}
\tolerance=750
\hfuzz=15pt
\hbadness=750
\setlength{\emergencystretch}{15pt}
\setlength{\parindent}{0cm}
\setlength{\parskip}{3ex plus 2ex minus 2ex}
\makeatletter
\renewcommand{\paragraph}{%
  \@startsection{paragraph}{4}{0ex}{-1.0ex}{1.0ex}{%
    \normalfont\normalsize\bfseries\SS@parafont%
  }%
}
\renewcommand{\subparagraph}{%
  \@startsection{subparagraph}{5}{0ex}{-1.0ex}{1.0ex}{%
    \normalfont\normalsize\bfseries\SS@subparafont%
  }%
}
\makeatother

% Headers & footers
\usepackage{fancyhdr}
\pagestyle{fancyplain}
\renewcommand{\sectionmark}[1]{%
  \markright{\thesection\ #1}%
}
\fancyhead[LE]{\fancyplain{}{\bfseries\thepage}}
\fancyhead[CE]{\fancyplain{}{}}
\fancyhead[RE]{\fancyplain{}{\bfseries\leftmark}}
\fancyhead[LO]{\fancyplain{}{\bfseries\rightmark}}
\fancyhead[CO]{\fancyplain{}{}}
\fancyhead[RO]{\fancyplain{}{\bfseries\thepage}}
\fancyfoot[LE]{\fancyplain{}{}}
\fancyfoot[CE]{\fancyplain{}{}}
\fancyfoot[RE]{\fancyplain{}{\bfseries\scriptsize \copyright{} 2005-2018 H\"orTech gGmbH, Oldenburg }}
\fancyfoot[LO]{\fancyplain{}{\bfseries\scriptsize \copyright{} 2005-2018 H\"orTech gGmbH, Oldenburg }}
\fancyfoot[CO]{\fancyplain{}{}}
\fancyfoot[RO]{\fancyplain{}{}}

% Indices & bibliography
\usepackage{natbib}
\usepackage{tocloft}
\setcounter{tocdepth}{2}
\setcounter{secnumdepth}{4}
\addtolength{\cftsubsecnumwidth}{5pt}
\makeindex

% Custom commands
\newcommand{\clearemptydoublepage}{%
  \newpage{\pagestyle{empty}\cleardoublepage}%
}

\usepackage{caption}
\captionsetup{labelsep=space,justification=centering,font={bf},singlelinecheck=off,skip=4pt,position=top}

%===== C O N T E N T S =====

\begin{document}

\MHAtitle{The Graphical User Interface Control Manual}
\newpage
\MHAcopyright{}
\newpage
\tableofcontents
\newpage
\renewcommand{\leftmark}{\rightmark}
\pagenumbering{arabic}

%--- Begin generated contents ---

\section{Introduction}

The H\"{o}rtech Open Master Hearing Aid (\mha{}) is an algorithm 
development and evaluation platform.
%
The real-time system can be configured to behave as a hearing device.
%
The program 'mhacontrol' provides a control surface and fitting
interface.
%
This interface is intended to be usable without technical insight into
the underlying algorithms.

This manual will describe the dialogs of the control surface and
explains typical work flows.

\section{Starting 'mhacontrol'}

The control surface 'mhacontrol'\index{mhacontrol} is a \Octave{}/\Matlab{}
application.
%
We assume that an instance of \mha{} is already running (see README.md and \EngineerManual{} for 
details). To start the control surface, please start \Octave{}/\Matlab{}, change into the 
directory with the application files (assuming your working directory to be the 
main directory of \mha{}, this should be '\MatlabDir{}'), and type
%
\begin{verbatim}
mhacontrol hostname
\end{verbatim}
%
\verb!hostname! is the hostname\index{hostname} or IP address\index{IP
  address} of the \mha{}.
%
If the \mha{} runs on the same host as \Octave{}/\Matlab{} the hostname or IP address
can be omitted.
%
If the \mha{} and \Octave{}/\Matlab{} do not run on the same host, please make sure
that the firewall allows access to port\index{port}\index{firewall}
33337 (or the port used by the actual \mha{} configuration), and that the
\mha{} on the remote host is started listening on the appropriate network
device (e.g., \verb!mha -i 0.0.0.0!).
%
A TCP network connection between \mha{} and 'mhacontrol' is required.

Please note that the control tool will overwrite many values of the
\mha{} configuration files with values stored in the configuration of
mhacontrol (stored in \Octave{}/\Matlab{} files on the control PC).
%
This primarily affects the client data, i.e. configuration of dynamic
compression and fine-tuning, and the calibration parameters.
%
If the MHAIOJack audio backend is used, the Jack connections might be
changed by the control surface.

\section{Dialogs}

\subsection{The main screen 'mhacontrol'}\index{mhacontrol}\index{main screen}

The main screen (\figref{mhacontrol_main.png}) contains buttons for all control applications the GUI currently provides:
%
In the upper right area is a panel with starter buttons for typically
used actions.
%
If a dynamic compression is available, a fitting interface (see
Section \ref{sec:fitting}) can be opened through the 'Fitting' button.
%
A level meter can be opened by clicking 'Levels' (see Section
\ref{sec:levels}).

Below the 'Actions' panel is the 'Administration'
panel\index{administration} with start buttons for actions which are
needed only for the maintaining of the \mha{} setup.
%
%These are 'Audio connections' (Section \ref{sec:audioconnections}) for
%a selection of input and output sound card connections.
%
The generic \mha{} configuration browser can be reached through 'Advanced
control' (see Section \ref{sec:advanced}).

\MHAfigure[][0.8\linewidth]{Main screen}{mhacontrol_main.png}

\subsection{Client selection and fitting}\label{sec:fitting}

\subsubsection*{Client database}

The 'mhacontrol' software contains a simple database for client management\index{client database}\index{database}.
%
Client entries can be created and modified in the client database
dialog, see \figref{mhacontrol_client_database.png}.
%
A client entry consists of first- and lastname, birthday, a unique
client ID\index{client ID}, and an arbitrary number of
audiograms\index{audiogram}.
%
In the top left panel a list of existing clients is shown.
%
A client can be selected for editing an audiogram entry by a single
mouse click on the list entry.
%
With the buttons 'New client' and 'Edit client', the client data can
be entered or edited.

\MHAfigure{Client database dialog.}{mhacontrol_client_database.png}

On the right side of the dialog audiogram data can be managed.
%
Available audiograms are listed in the top right panel.
%
A new audiogram can be created by either manually entering the
audiogram data (see \figref{mhacontrol_audiogram_entry.png}, button
'Edit').
%
The selected audiogram can be printed using the 'Print' button.

\subsubsection*{Audiogram editor}

An audiogram can be manually edited using the audiogram editor (see
\figref{mhacontrol_audiogram_entry.png}).\index{audiogram editor}\index{hearing threshold}\index{HTL}\index{UCL}
%
In the upper half of the dialog, the hearing threshold and
uncomfortable level can be entered for the right and left ear.
%
To skip a frequency, a value of 'inf' or 'nan' can be entered.
%
Below, on the left side, an audiogram ID can be entered.
%
The audiogram editor can be closed by hitting the 'Finish' button.
%
The current audiogram is displayed in the lower right side of the
dialog.

\MHAfigure{Audiogram editor.}{mhacontrol_audiogram_entry.png}

\subsubsection*{Fitting dialog}

After hitting 'OK' the fitting dialog opens (see \figref{mhacontrol_fitting_tool.png}). It is divided into three columns. The middle column contains control
elements to select a gain prescription rule and create a first
fit. Below is a box for preset management. A preset contains all
settings which can be made in this dialog, including gain prescription
rule and fine-tuning, for a specific client and algorithm.  The
columns on the left and right side of the screen contain the
ear-specific settings for the right and left ear, respectively. In the
graph the nominal target level of a (long term average speech spectrum) LTASS at three different levels is
shown with thick solid lines, calculated in third-octave bands. The
thin dashed lines mark the corresponding input level. The shaded areas
mark the ISO-226 normal hearing threshold (gray) and the threshold of
the selected subject (green). Below the graph controls for fine-tuning
can be found: An additional gain and a gain limitation can be
configured in frequency bands, by selecting \verb!gain! or
\verb!maxgain! in the drop-down menu.

To create a first fit, first select the appropriate gain prescription
rule from the list. In \mha{} configurations with a unilateral
compressor, also the ear to be fitted must be selected\footnote{The
  selection of an ear in the fitting dialog does not affect any signal
  routing. A correct mapping of \mha{} input and output channels to the
  correct ear must be assured.}. Pressing the `Create First Fit'
button will create a first fit, upload it to the \mha{} and store it as a
preset in the preset list. The fine-tuning parameters are set to
default values.  A first fit can not be overwritten by a fine-tuned
preset.  Creating a new first fit (e.g. after selecting another
audiogram) will overwrite any previous first fit for the selected gain
rule.

\MHAfigure[][\linewidth]{Fitting dialog.}{mhacontrol_fitting_tool.png}

\subsection{Level meter}\label{sec:levels}\index{level meter}\index{broadband level}\index{gain}\index{input level}\index{output level}

The level meter (\figref{mhacontrol_levels.png}) contains two areas.
%
In the upper panel, the broad band level at input and output is shown.
%
Below that panel, the time constant for RMS level averaging can be
adjusted (broad band level meter only).

\MHAfigure{Level and gain meter.}{mhacontrol_levels.png}
%
%\subsection{Audio connections}\label{sec:audioconnections}\index{connections}\index{JACK}\index{audio backend}
%
%The 'mhacontrol' application can provide a dialog for connecting input
%and output channels to specific hardware audio channels, see
%\figref{mhacontrol_audio_connections.png} for a screenshot (this
%requires the JACK audio backend).
%
%Each \mha{} input channel can be connected to a hardware input, and
%each output channel can be connected to a hardware output.
%
%Connections can be saved in presets that are listed in the left panel of the connection dialog.
%
%A preset can be selected by clicking on its entry in the list.
%
%Please note that in the connection dialog only connections to hardware
%ports are supported.
%
%If you need connections to software ports, please use JACK connection
%tools on the \mha{} computer.
%
%
%\MHAfigure[][0.8\linewidth]{Audio connection manager.}{mhacontrol_audio_connections.png}

\subsection{Advanced control}\label{sec:advanced}\index{advanced control}\index{mhagui\_generic}

A generic parameter control interface to the \mha{} is available. See
\figref{mhacontrol_advanced.png} for a typical root level dialog.
%
A description of this generic interface can be found in the \mha{}
manual.

\MHAfigure[1.5\linewidth]{Generic \mha{} parameter interface.}{mhacontrol_advanced.png}

\bibliography{MHA}

\printindex

\end{document}

% LocalWords:  audiogram \mha{} LTASS

%%% Local Variables: 
%%% mode: latex
%%% TeX-master: "openMHA_application_manual"
%%% indent-tabs-mode: nil
%%% coding: utf-8-unix
%%% End:
