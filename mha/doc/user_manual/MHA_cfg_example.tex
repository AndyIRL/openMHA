%%% This file is part of the Open HörTech Master Hearing Aid (openMHA)
%%% Copyright © 2017 2018 HörTech gGmbH
%%%
%%% openMHA is free software: you can redistribute it and/or modify
%%% it under the terms of the GNU Affero General Public License as published by
%%% the Free Software Foundation, version 3 of the License.
%%%
%%% openMHA is distributed in the hope that it will be useful,
%%% but WITHOUT ANY WARRANTY; without even the implied warranty of
%%% MERCHANTABILITY or FITNESS FOR A PARTICULAR PURPOSE.  See the
%%% GNU Affero General Public License, version 3 for more details.
%%%
%%% You should have received a copy of the GNU Affero General Public License, 
%%% version 3 along with openMHA.  If not, see <http://www.gnu.org/licenses/>.

\section{A simple example configuration and how to start it}%
\label{sec:scenarios}%
\index{example configuration}%
\index{configuration!example}%
\index{configuration file}%

In this section, an example is shown on how to configure and start the
\mha{}. A simple algorithm is designed, which implements a 
multi-band dynamic range compressor. Then examples are given on
how to start this algorithm in different situations (\mhad{}, \Matlab{} processing).

\subsection{Dynamic compressor algorithm}%
\label{sec:dyncmp}%

This example is meant to serve as an illustrative case and a more complete 
configuration file with detailed explanations for dynamic compression is given 
in the file \newline \emph{dynamiccompression.cfg}.
%
In this example the host plugin, the plugin that loads other plugins, 
will be \verb!transducers!. This plugin is used for calibrating the 
input and output signals.
%
The input signal will be split in two frequency bands, using the \mha{}
plugins \verb!fftfilterbank!. Only spectral processing is required, thus
the plugin \verb!overlapadd! will be
used by the framework. The corresponding configuration file for this
plugin is \emph{example\_dc.cfg}, which can be found in
the directory \verb!mha/examples/01-dynamic-compression!.

%\mha{} configuration files usually start with an identification line:
%\begin{verbatim}
%#[MHAVersion 4.4]
%\end{verbatim}
%Please note that since version 4.4, the first line in configuration
%files is not treated as an identification line, thus a comment
%character is required before the version number.
We first configure general parameters, such as the number of input 
channels, fragment size and the sampling rate:
\begin{verbatim}
nchannels_in = 2
fragsize = 64
srate = 44100
\end{verbatim}
%
We then load the host plugin and specify the input-output backend, which 
in this case we will be using audio files:
\begin{verbatim}
mhalib = transducers
iolib = MHAIOFile
\end{verbatim}
%
Next, the host plugin \verb!transducers! loads the \verb!overlapadd! plugin 
to perform the conversion between time and spectral domains:
\begin{verbatim}
mha.plugin_name = overlapadd
\end{verbatim}
%
Also, we specify the audio channel-specific ``peak levels'' for input and output sound 
signals in dB:
\begin{verbatim}
mha.calib_in.peaklevel = [90 90]
mha.calib_out.peaklevel = [90 90]
\end{verbatim}
In the next few lines, we configure the parameters of the overlap-add
method. Recall that a fragment size of 64 samples is used. Zeros will be padded
at both ends to avoid circular aliasing.
\begin{verbatim}
mha.overlapadd.fftlen = 256
mha.overlapadd.wnd.len = 128
\end{verbatim}
%
Within the overlapadd framework, we process a sequence of plugins
by setting the \verb!mhachain! plugin:
\begin{verbatim}
mha.overlapadd.plugin_name = mhachain
\end{verbatim}
%
A single chain will be processed, with a frequency grouping into bands, a dynamic 
compressor and a filterbank re-synthesis at the output:
\begin{verbatim}
# list of plugins
mha.overlapadd.mhachain.algos = [ ... 
fftfilterbank ... 
dc ... 
combinechannels ...
]
\end{verbatim}
In the next step, the filter bank will be configured with two only 
frequency bands. 
\begin{verbatim}
mha.overlapadd.mhachain.fftfilterbank.f = [200 2000]
\end{verbatim}
%
The dynamic compression algorithm measures the input sound level in 
each frequency band and determines the gain to be applied by looking 
it up in a gain table. The gain table has one row of gains for each 
frequency band of the left audio channel, followed by the same for 
the right audio channel. In our case, the number of rows of the 
matrix will be 4.
%
The first element in each row (i.e., taken together, the first
column) specifies the gain in dB to be applied if the input level in
the respective frequency band is equal to the value of the \verb!gtmin!
element given for that respective band.
%
The following elements in each row specify the gains in dB to be
applied for other input values, where the input level difference
between the individual elements in each row of the matrix is
determined by the value of \verb!gtstep! for the respective 
band. The dynamic compressor also employs a common attack/decay low-pass filter
to determine the input level (for in-depth explanation of the parameters 
listed below, see file \emph{dynamiccompression.cfg}):
\begin{verbatim}
# gaintable data in dB gains
mha.overlapadd.mhachain.dc.gtdata = [[10 -10 -30];...
                                     [20 -25 -50];...
                                     [10 -10 -30];...
                                     [20 -25 -50]]
# input level for first gain entry in dB SPL
mha.overlapadd.mhachain.dc.gtmin = [0]
# level step size in dB
mha.overlapadd.mhachain.dc.gtstep = [40]
# attack time constant in s
mha.overlapadd.mhachain.dc.tau_attack = [0.02]
# decay time constant in s
mha.overlapadd.mhachain.dc.tau_decay = [0.1]
\end{verbatim}
%
We also have to include in the dynamic compressor plugin, the name 
of the filter bank plugin, used to extract frequency information
\begin{verbatim}
# Name of fftfilterbank plugin.  Used to extract frequency information.
mha.overlapadd.mhachain.dc.fb = fftfilterbank
mha.overlapadd.mhachain.dc.chname = fftfilterbank_nchannels
\end{verbatim}
%
The number of output channels after that plugin \verb!combinechannels! 
gives out after recombining the signal from different frequency 
bands should be specified
\begin{verbatim}
mha.overlapadd.mhachain.combinechannels.outchannels = 2
\end{verbatim}
%
Finally, we specify the input and output audio files to be processed as 
follows:
\begin{verbatim}
io.in = 1speaker_diffNoise_2ch.wav
io.out = 1speaker_diffNoise_2ch_OUT.wav
\end{verbatim}
The configuration we described so far is provided with this release 
in the file \emph{example\_dc.cfg}.
Now that we created a complete \mha{} hearing aid
algorithm configuration file, In the next step, we want to start an \mha{}
framework with this configuration.
%
In the terminal, change the working directory to the
\verb!mha/examples/01-dynamic-compression! 
directory of the \mha{}.
All binaries of the \verb!mha! should be in the executable search
\verb!PATH! (environment variable, see \emph{README.md}).
Then, MHA processing can be started with
\begin{verbatim}
mha ?read:example_dc.cfg cmd=start cmd=quit
\end{verbatim}
%
This will process the file \emph{1speaker\_diffNoise\_2ch.wav} and output 
\emph{1speaker\_diffNoise\_2ch\_OUT.wav}.
%\subsection{Using the \mhad{} and \Matlab{}}

%The \mhad{} configuration interface can be easily accessed from the
%\Matlab{} prompt through a generic graphical user interface,
%'mhagui\_generic' (see also \secpageref{sec:mhagui_generic}).
%This interface works on \Linux{} and \Windows{}. To start the \mhad{} on \Linux{}, type
%\begin{verbatim}
%MHA_LIBRARY_PATH=./
%LD_LIBRARY_PATH=./
%mha
%\end{verbatim}
%in the MHA 'bin' directory; on \Windows{} just double-click the file
%'mha.exe'. Then start \Matlab{}, and add the MHA matlab directory to
%your \Matlab{} search path using 'addpath'. Finally, before opening the user interface, add \verb!mhactl_java.jar! into your \Matlab{} classpath. To open the user interface, type
%\begin{verbatim}
%mhagui_generic
%\end{verbatim}
%at the \Matlab{} prompt.

\subsection{Starting the \mhad{} with JACK}%
\label{sec:example_jack}%
\index{JACK}%
\index{framework configuration}%
\index{configuration!framework}%

We use the configuration settings described in the previous section 
and use the JACK server as an audio backend. In the previous section we set the number of 
input channels, fragment size and sampling rate of the framework. 
When using JACK the fragment size
and sampling rate have to match those values used by the JACK server.
Here, we assume that the JACK server runs with a fragment size
of 64 samples, so that the overlap in the overlap-add method is
50\%. Later we will show how to use a double buffer for those
situations, where it is not possible to start JACK with the desired
fragment size. 
%\begin{verbatim}
%#[MHAVersion 4.4]
%nchannels_in = 2
%fragsize = 64
%srate = 48000
%\end{verbatim}
In the previous section we have specified which libraries to use 
(plugins and the IO backend, which was \verb!MHAIOFile!). In this section 
we want to change IO backend to JACK. 
%
To do this, we modify the following line:
\begin{verbatim}
# IO plugin library name
iolib = MHAIOJack
\end{verbatim}
%
As in the previous section, it is assumed, that the current working directory is the main \mha{} directory, and the 
binaries can be found in \verb!./bin/!, i.e.\ the environment variables 
MHA\_LIBRARY\_PATH and LD\_LIBRARY\_PATH are configured
properly (see \verb!README.md!).
%
Please note that these environment variables have to be
terminated with a slash.
%
%No extension is required in the configuration of plugin names.
%\begin{verbatim}
%iolib = MHAIOJack
%mhalib = overlapadd
%\end{verbatim}
Finally, we can connect the JACK client to the hardware input and
output ports and load the algorithm configuration file, for example:
\begin{verbatim}
io.con_in = [system:capture_5 system:capture_6]
io.con_out = [system:playback_1 system:playback_2]
\end{verbatim}
Please replace the port names by the ports you want to connect
to.
%

Another alternative to start the \mhad{} is from \Matlab{}. On Linux, one can call
\begin{verbatim}
[errcode, pid] = system('mha & echo $!')
\end{verbatim}
(assuming, that the directory containing the MHA binaries is included
in the system path and in the MHA\_LIBRARY\_PATH environment
variable. Please note that in Octave the environment variables 
may have to be set again.).
%
The variable \verb!pid! then contains the process id of the \mhad{} process as text.
%
The \mha{} configuration and control
input is accepted over a network connection. 
%
When no command line parameters are given, the default port number
33337 and the loopback network interface 127.0.0.1 is used, i.e., only
connections from the local host are accepted (see section
\ref{sec:linuxmhaserver} on page \pageref{sec:linuxmhaserver} for
details).
%
To enter \mha{} commands, please start a network client, e.g.\ Netcat, to
open a MHA console:
\begin{verbatim}nc localhost 33337\end{verbatim}

To read the framework configuration file, please type
\begin{verbatim}
?read:mha/configurations/example_dc_live.cfg
\end{verbatim}
followed by the return key. If everything went well, the MHA will print
\verb!(MHA:success)!. In case of an error message \verb!(MHA:failure)!, MHA 
will also indicate the line containing the error. You need to correct this 
error using an editor you prefer and  reload it. If you receive an error 
message, \verb!(mha_parser) The variable is locked!, you need to close the 
\mhad{} and relaunch it. If the JACK server was not started yet, this is the
right moment to start your JACK server with the correct settings. One can use 
for example the \verb!QjackCtl! client to set the correct settings.
%e.g.:
%\begin{verbatim}
%jackd -d alsa -r 44100 -p 64
%\end{verbatim}
Please make sure that the fragment size and sample rate of the JACK
sound server matches the MHA fragment size (see below if it
doesn't). After having successfully started the JACK server, the MHA
can be started by typing
\begin{verbatim}
cmd = start
\end{verbatim}
at the MHA console. The processing can be stopped at any time by
typing \verb!cmd = stop!.

Now, we want to access the variables of the algorithm. The easiest way
is to type \verb!?!, followed by the return key, in the console. This
will show the complete MHA configuration, including all framework
variables and plugin configuration. Usually, this
produces so much output, that the console has to be scrolled to see
the complete information. If only a subset or a single variable is of
interest, the prefix of that subset or variable can be put before the
\verb!?!, e.g.\ all variables of the processing chain can be reached
by typing \newline\verb!mha.overlapadd.mhachain?!, the gaintable data in dB gains
by typing \newline\verb!mha.overlapadd.mhachain.dc.gtdata?!. All monitor variable contents can
be stored into the file "example.mon" by typing
\verb!?savemons:example.mon!. The \mhad{} will be closed
by typing \verb!cmd = quit!.

\paragraph{Adjusting the fragment size}%
\index{fragment size}%
\index{double buffering}%

If the required fragment size is not supported by the audio hardware,
double buffering can be used in the MHA frameworks. We assume now,
that the JACK server was started with a fragment size of 256 samples at
a sampling rate of 44.1 kHz. A fragment size of 64 samples in the algorithm 
can be reached by inserting a double buffer plugin between the framework and the algorithm. This is
done by replacing the MHA library \verb!overlapadd! by the double
buffer plugin \verb!db!, which will load \verb!overlapadd! as a
client. In the framework configuration in \verb!fwex1.cfg!, the line
\verb!mhalib = overlapadd! will be replaced by
\begin{verbatim}
mhalib = db
mha.plugin_name = overlapadd
mha.fragsize = 64
\end{verbatim}
The fragment size of the framework will be set to that of the JACK
server, so please replace \verb!fragsize = 64! on the top level by
\verb!fragsize = 256!. The hierarchy of layers now changes (see 
file \emph{example\_dc\_live\_double.cfg}, where between the framework 
and \verb!overlapadd! lies the \verb!db! plugin.
Similarly, the variable contents of the processing chain can now be
queried by typing \verb!mha.db.overlapadd.mhachain?!.

\subsection{Start the MHA for real-time processing with \Octave{}/\Matlab{} from Linux}%
\index{Matlab}%

The configuration\index{framework
configuration}\index{configuration!framework} of the \mhad{} can be
read at startup time by adding an \mha{} configuration language command:
%
\begin{verbatim}
./bin/mha ?read:mha/configurations/example_dc.cfg cmd=start
\end{verbatim}
%
We assume, that your \Matlab{} process is running on the same host and
as the same user as the \mha{}, and that the \mhad{} runs with the
default port number 33337.
%
The \mha{} \Matlab{} tools directory has to be in the \Matlab{} path.
%
Now, communication with the configuration interface of the MHA is
possible through MHA \Matlab{} functions (see \secpageref{sec:matlab}
for a detailed documentation):
%
First, create a MHA connection handle for the \Matlab{} tools by
typing
\begin{verbatim}
h = struct( 'port', 33337, 'host', 'localhost' );
\end{verbatim}
at the \Matlab{} prompt. Then this handle can be used to connect to
the MHA:
%
\begin{verbatim}
result = mha_get( h, '' );
\end{verbatim}
%
The complete MHA configuration hierarchy is converted into a \Matlab{}
'struct' variable.
%
When the \mha{} processing is not needed any more, the MHA can be shut
down by calling
%
\begin{verbatim}
mha_set( h, 'cmd', 'quit' );
\end{verbatim}


%%% Local Variables: 
%%% mode: latex
%%% TeX-master: "openMHA_application_manual"
%%% indent-tabs-mode: nil
%%% coding: utf-8-unix
%%% End:
