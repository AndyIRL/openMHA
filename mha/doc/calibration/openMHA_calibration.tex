%%% This file is part of the Open HörTech Master Hearing Aid (openMHA)
%%% Copyright © 2018 HörTech gGmbH
%%%
%%% openMHA is free software: you can redistribute it and/or modify
%%% it under the terms of the GNU Affero General Public License as published by
%%% the Free Software Foundation, version 3 of the License.
%%%
%%% openMHA is distributed in the hope that it will be useful,
%%% but WITHOUT ANY WARRANTY; without even the implied warranty of
%%% MERCHANTABILITY or FITNESS FOR A PARTICULAR PURPOSE.  See the
%%% GNU Affero General Public License, version 3 for more details.
%%%
%%% You should have received a copy of the GNU Affero General Public License, 
%%% version 3 along with openMHA.  If not, see <http://www.gnu.org/licenses/>.

% Latex header for doxygen 1.8.11
\documentclass[11pt,a4paper,twoside]{article}

% Packages required by doxygen
\usepackage{fixltx2e}
\usepackage{calc}
\usepackage{../openMHAdoxygen}
\setlength{\headheight}{13.6pt}
\usepackage[export]{adjustbox} % also loads graphicx
\usepackage{graphicx}
\usepackage[utf8]{inputenc}
\usepackage{makeidx}
\usepackage{multicol}
\usepackage{multirow}
\PassOptionsToPackage{warn}{textcomp}
\usepackage{textcomp}
\usepackage[nointegrals]{wasysym}
\usepackage[table]{xcolor}

% Font selection
\usepackage[T1]{fontenc}
\usepackage{helvet}
\usepackage{courier}
\usepackage{amssymb}
\usepackage{sectsty}
\renewcommand{\familydefault}{\sfdefault}
\allsectionsfont{%
  \fontseries{bc}\selectfont%
  \color{darkgray}%
}
\renewcommand{\DoxyLabelFont}{%
  \fontseries{bc}\selectfont%
  \color{darkgray}%
}
\newcommand{\+}{\discretionary{\mbox{\scriptsize$\hookleftarrow$}}{}{}}

% Page & text layout
\usepackage{geometry}
\geometry{%
  a4paper,%
  top=2.5cm,%
  bottom=2.5cm,%
  left=2.5cm,%
  right=2.5cm%
}
\tolerance=750
\hfuzz=15pt
\hbadness=750
\setlength{\emergencystretch}{15pt}
\setlength{\parindent}{0cm}
\setlength{\parskip}{3ex plus 2ex minus 2ex}
\makeatletter
\renewcommand{\paragraph}{%
  \@startsection{paragraph}{4}{0ex}{-1.0ex}{1.0ex}{%
    \normalfont\normalsize\bfseries\SS@parafont%
  }%
}
\renewcommand{\subparagraph}{%
  \@startsection{subparagraph}{5}{0ex}{-1.0ex}{1.0ex}{%
    \normalfont\normalsize\bfseries\SS@subparafont%
  }%
}
\makeatother

% Headers & footers
\usepackage{fancyhdr}
\pagestyle{fancyplain}
\renewcommand{\sectionmark}[1]{%
  \markright{\thesection\ #1}%
}
\fancyhead[LE]{\fancyplain{}{\bfseries\thepage}}
\fancyhead[CE]{\fancyplain{}{}}
\fancyhead[RE]{\fancyplain{}{\bfseries\leftmark}}
\fancyhead[LO]{\fancyplain{}{\bfseries\rightmark}}
\fancyhead[CO]{\fancyplain{}{}}
\fancyhead[RO]{\fancyplain{}{\bfseries\thepage}}
\fancyfoot[LE]{\fancyplain{}{}}
\fancyfoot[CE]{\fancyplain{}{}}
\fancyfoot[RE]{\fancyplain{}{\bfseries\scriptsize \copyright{} 2005-2018 H\"orTech gGmbH, Oldenburg }}
\fancyfoot[LO]{\fancyplain{}{\bfseries\scriptsize \copyright{} 2005-2018 H\"orTech gGmbH, Oldenburg }}
\fancyfoot[CO]{\fancyplain{}{}}
\fancyfoot[RO]{\fancyplain{}{}}

% Indices & bibliography
\usepackage{natbib}
\usepackage{tocloft}
\setcounter{tocdepth}{2}
\setcounter{secnumdepth}{4}
\addtolength{\cftsubsecnumwidth}{5pt}
\makeindex

% Custom commands
\newcommand{\clearemptydoublepage}{%
  \newpage{\pagestyle{empty}\cleardoublepage}%
}

\usepackage{caption}
\captionsetup{labelsep=space,justification=centering,font={bf},singlelinecheck=off,skip=4pt,position=top}

%===== C O N T E N T S =====

\begin{document}

\MHAtitle{The Calibration Manual}
\newpage
\MHAcopyright{}
\newpage
\tableofcontents
\newpage
\renewcommand{\leftmark}{\rightmark}
\pagenumbering{arabic}

\section{Introduction}

The H\"{o}rtech Open Master Hearing Aid (\mha{}) is a software platform
for algorithm development and evaluation.
%
For certain aspects of algorithm evaluation, correct calibration of the
test setup may be a requirement.
%
Since \mha{}, being a software platform, does not include hardware
(like hearing aid housings, microphones, and receivers),
users of the \mha{} are expected to supplement suitable hardware
to pick up and produce sound at the users' ears.
%
Users then face the challenge of how to correctly configure the software
to calibrate the system.
%
This guide is a collection of best practices that the authors and users
of \mha{} employ to ensure their setups are calibrated correctly.
%
We will extend this guide with new procedures as we add them to the
openMHA, and as we receive user contributions in the field of
calibration.
%
\mha{} users should be aware that calibration is an involved topic.
%
Users need professional, specialized acoustic measurement equipment in
order to be able to perform accurate measurements, and understand
how to use their equipment.
%
Understanding how to calibrate commercial hearing aids is a requirement
for using this guide.
%
The book "Hearing Aids" by Harvey Dillon contains a suitable treatment
of the topic in chapter 4 of its second edition (2012).
%
We cannot guarantee that the procedures presented here are free of
errors.
%
If you discover something that is wrong, please get in contact with
the \mha{} project, e.g. by filing an issue on the github project.
%
We will do our best to improve the documentation and the tools that we
provide as we go along.

\section{Calibrating hearing aid outputs}

This section describes how to measure the frequency response of a
class of custom hearing aid output hardware, and how to use frequency
responses to calibrate individual devices of this class.

%

\subsection{Considerations regarding calibration of custom hardware used as hearing aid outputs}

Hearing aid receivers typically have a low impedance and a non-flat frequency response.
%
When attaching hearing aid receivers to the output of a computer's
sound card, care has to be taken to ensure that the sound card can
actually drive the receiver, i.e.\ provide the necessary current
without voltage drops.
%
Otherwise serious non-linearities will result.
%
Consider the application of a suitable amplifier to drive your output hardware.

A suitable acoustic coupler has to be used to connect the measurement
microphone to the hearing aid receiver.
%
The coupler will simulate the mechanical properties similar to
a real ear canal.
%
The coupler will also influence how the measurement microphone of your
measurement equipment has to be calibrated.
%
Please refer to Dillon and for the manuals of your measurement
equipment for details.

When you calibrate custom hearing aid hardware, you always calibrate
the complete output chain in combination, consisting of
%
the individual sound card, any amplifier that you employ, and the
hearing aid receiver.

Hearing aids need to produce ear canal sound levels, not free field
sound levels.
%
Please refer to the Real-Ear-Unaided-Gain in Dillon.

When calibrating hearing aid devices, you need to make a choice
whether you are ok with calibrating the device for an average ear or
for the individual ear of a specific subject.
%
We will describe only the average calibration method for now, but this
may change in future.

\subsection{Measuring the frequency response for a class of hearing aid output software with openMHA}

\subsection{Using the measured frequency response to calibrate a hearing aid output device}

\bibliography{MHA}

\printindex

\end{document}

% LocalWords:  audiogram \mha{} LTASS

%%% Local Variables: 
%%% mode: latex
%%% TeX-master: "openMHA_application_manual"
%%% indent-tabs-mode: nil
%%% coding: utf-8-unix
%%% End:
