%%% This file is part of the Open HörTech Master Hearing Aid (openMHA)
%%% Copyright © 2018 HörTech gGmbH
%%%
%%% openMHA is free software: you can redistribute it and/or modify
%%% it under the terms of the GNU Affero General Public License as published by
%%% the Free Software Foundation, version 3 of the License.
%%%
%%% openMHA is distributed in the hope that it will be useful,
%%% but WITHOUT ANY WARRANTY; without even the implied warranty of
%%% MERCHANTABILITY or FITNESS FOR A PARTICULAR PURPOSE.  See the
%%% GNU Affero General Public License, version 3 for more details.
%%%
%%% You should have received a copy of the GNU Affero General Public License, 
%%% version 3 along with openMHA.  If not, see <http://www.gnu.org/licenses/>.

% Latex header for doxygen 1.8.11
\documentclass[11pt,a4paper,twoside]{article}

% Packages required by doxygen
\usepackage{fixltx2e}
\usepackage{calc}
\usepackage{../openMHAdoxygen}
\setlength{\headheight}{13.6pt}
\usepackage[export]{adjustbox} % also loads graphicx
\usepackage{graphicx}
\usepackage[utf8]{inputenc}
\usepackage{makeidx}
\usepackage{multicol}
\usepackage{multirow}
\PassOptionsToPackage{warn}{textcomp}
\usepackage{textcomp}
\usepackage[nointegrals]{wasysym}
\usepackage[table]{xcolor}

% Font selection
\usepackage[T1]{fontenc}
\usepackage{helvet}
\usepackage{courier}
\usepackage{amssymb}
\usepackage{sectsty}
\renewcommand{\familydefault}{\sfdefault}
\allsectionsfont{%
  \fontseries{bc}\selectfont%
  \color{darkgray}%
}
\renewcommand{\DoxyLabelFont}{%
  \fontseries{bc}\selectfont%
  \color{darkgray}%
}
\newcommand{\+}{\discretionary{\mbox{\scriptsize$\hookleftarrow$}}{}{}}

% Page & text layout
\usepackage{geometry}
\geometry{%
  a4paper,%
  top=2.5cm,%
  bottom=2.5cm,%
  left=2.5cm,%
  right=2.5cm%
}
\tolerance=750
\hfuzz=15pt
\hbadness=750
\setlength{\emergencystretch}{15pt}
\setlength{\parindent}{0cm}
\setlength{\parskip}{3ex plus 2ex minus 2ex}
\makeatletter
\renewcommand{\paragraph}{%
  \@startsection{paragraph}{4}{0ex}{-1.0ex}{1.0ex}{%
    \normalfont\normalsize\bfseries\SS@parafont%
  }%
}
\renewcommand{\subparagraph}{%
  \@startsection{subparagraph}{5}{0ex}{-1.0ex}{1.0ex}{%
    \normalfont\normalsize\bfseries\SS@subparafont%
  }%
}
\makeatother

% Headers & footers
\usepackage{fancyhdr}
\pagestyle{fancyplain}
\renewcommand{\sectionmark}[1]{%
  \markright{\thesection\ #1}%
}
\fancyhead[LE]{\fancyplain{}{\bfseries\thepage}}
\fancyhead[CE]{\fancyplain{}{}}
\fancyhead[RE]{\fancyplain{}{\bfseries\leftmark}}
\fancyhead[LO]{\fancyplain{}{\bfseries\rightmark}}
\fancyhead[CO]{\fancyplain{}{}}
\fancyhead[RO]{\fancyplain{}{\bfseries\thepage}}
\fancyfoot[LE]{\fancyplain{}{}}
\fancyfoot[CE]{\fancyplain{}{}}
\fancyfoot[RE]{\fancyplain{}{\bfseries\scriptsize \copyright{} 2005-2018 H\"orTech gGmbH, Oldenburg }}
\fancyfoot[LO]{\fancyplain{}{\bfseries\scriptsize \copyright{} 2005-2018 H\"orTech gGmbH, Oldenburg }}
\fancyfoot[CO]{\fancyplain{}{}}
\fancyfoot[RO]{\fancyplain{}{}}

% Indices & bibliography
\usepackage{natbib}
\usepackage{tocloft}
\setcounter{tocdepth}{2}
\setcounter{secnumdepth}{4}
\addtolength{\cftsubsecnumwidth}{5pt}
\makeindex

% Custom commands
\newcommand{\clearemptydoublepage}{%
  \newpage{\pagestyle{empty}\cleardoublepage}%
}

\usepackage{caption}
\captionsetup{labelsep=space,justification=centering,font={bf},singlelinecheck=off,skip=4pt,position=top}

%===== C O N T E N T S =====

\begin{document}

\MHAtitle{The Calibration Manual}
\newpage
\MHAcopyright{}
\newpage
\tableofcontents
\newpage
\renewcommand{\leftmark}{\rightmark}
\pagenumbering{arabic}

\section{Introduction}

The H\"{o}rtech Open Master Hearing Aid (\mha{}) is a software platform
for algorithm development and evaluation.
%
For certain aspects of algorithm evaluation, correct calibration of the
test setup may be a requirement.
%
Since \mha{}, being a software platform, does not include hardware
(like hearing aid housings, microphones, and receivers),
users of the \mha{} are expected to supplement suitable hardware
to pick up and produce sound at the users' ears.
%
Users then face the challenge of how to correctly configure the software
to calibrate the system.
%
This guide is a collection of best practices that the authors and users
of \mha{} employ to ensure their setups are calibrated correctly.
%
We will extend this guide with new procedures as we add them to the
openMHA, and as we receive user contributions in the field of
calibration.
%
\mha{} users should be aware that calibration is an involved topic.
%
Users need professional, specialized acoustic measurment equipment in
order to be able to perform accurate measurements, and understand
how to use their equipment.
%
Understanding how to calibrate commercial hearing aids is a requirement
for using this guide.
%
The book "Hearing Aids" by Harvey Dillon contains a suitable treatment
of the topic in chapter 4 of its second edition (2012).
%
We cannot guarantee that the procedures presented here are free of
errors.
%
If you discover something that is wrong, please get in contact with
the \mha{} project, e.g. by filing an issue on the github project.
%
We will do our best to improve the documentation and the tools that we
provide as we go along.

\section{Calibrating hearing aid outputs}

This section describes how to measure the frequency response of a
class of custom hearing aid output hardware, and how to use frequency
responses to calibrate individual devices of this class.

%

\subsection{Considerations regarding calibration of custom hardware used as hearing aid outputs}

Hearing aid receivers typically have a low impedance and a non-flat frequency response.
%
When attaching hearing aid receivers to the output of a computer's
sound card, care has to be taken to ensure that the sound card can
actually drive the receiver, i.e.\ provide the necessary current
without voltage drops.
%
Otherwise serious non-linearities will result.
%
Consider the application of a suitable amplifier to drive your output hardware.

A suitable acoustic coupler has to be used to connect the measurement
microphone to the hearing aid receiver.
%
The coupler will simulate the mechanical properties similar to
a real ear canal.
%
The coupler will also influence how the measurment microphone of your
measurement equipment has to be calibrated.
%
Please refer to Dillon and for the manuals of your measurement
equipment for details.

When you calibrate custom hearing aid hardware, you always calibrate
the complete output chain in combination, consisting of
%
the individual sound card, any amplifier that you employ, and the
hearing aid receiver.

Hearing aids need to produce ear canal sound levels, not free field
sound levels.
%
Please refer to the Real-Ear-Unaided-Gain (REUG) in Dillon.

When calibrating hearing aid devices, you need to make a choice
whether you are ok with calibrating the device for an average ear, or
for the individual ear of a specific subject.
%
We will describe only the average calibration method for now, but this
may change in future.

Calibration requires reliable measurement of acoustic levels.
%
This can only be achieved in a quiet environment.
%
Please perform all acoustic measurements in a suitable, sound-proof
booth or a similar environment.
%
If your calibration equipment supports it, consider to filter out very
low frequencies (e.g. < 20 Hz),
%
If you notice fluctuations in the measured level, e.g. resulting from
doors on other levels in your building, or from traffic outside,
consider moving to a better isolated location, or to off-peak hours.
%
To ensure that the influence of unavoidable noise is minimal,
acoustic calibration is performed at relatively levels around 80 dB.
%
If you cannot achieve these levels with your custom output hardware,
you should consider modifying the output hardware rather than
measuring a lower sound level.
%
We will be using the MHA to produce test tones of which we then
measure the acoustic levels.
%
When producing test tones with high frequencies, take into account
that the sound card will apply a low-pass filter for anti-aliasing
that affects already frequencies below the Nyquist frequency.

The calibration will run into practical limits, and will require
compromises.
%
E.g., hearing aid receivers typically produce significantly softer
sound levels for the same electric voltages at low frequencies.
%
Exact calibration requires filters that invert the frequency response
of the hearing aid output hardware, which would mean in this case,
amplify with a high gain at low frequencies.
%
A high enough gain at low frequencies will cause the output signal
amplitude to exceed the dynamic range of the sound card.
%
Since it is not possible to produce such a signal with a sound card,
the signal needs to be limited after the calibration filter is applied,
which will cause audible artifacts.
%
To avoid these situations, it is usually a better choice to avoid too
high amplification at the boundaries of the transmitted spectrum
and accept that the output levels are softer than intended by the
hearing aid signal processing algorithms in these areas.

\subsection{Measuring the frequency response for a class of custom hearing aid output software with openMHA with an IEC 711 coupler}

We describe here how to use a coupler according to IEC 60 711 to
measure the frequency response of custom hearing aid output hardware.
%
Compared to the 2cc coupler, the IEC 711 coupler has the advantage
that for an average ear, the Real-Ear-to-Coupler-Difference (RECD) is
so small that it does not need to be taken into account when doing
measurements to calibrate for the average ear.

Please assemble and calibrate your measurement equipment to perform
measurements with the IEC 711 coupler. Please refer to the
instructions of the equipment manufacturer on how to perform
this.
%
Please note that some calibrators require that you take correction
factors into account when calibrating your equipment for the IEC 711.

When we talk of a class of custom hearing aid output software, we
refer to an ensemble of output devices that use the same type of
receiver, the same type of amplifier, and the same type of soundcard.
%
Individual receivers, amplifiers, and soundcards will differ from each
other, even if they are of the same type.
%
You will have to decide for yourself what difference you are going to
accept and still consider the devices to be in the same class.

You will then measure the frequency responses of the output hardware
with the help of the openMHA:

\begin{itemize}
\item Start the jack sound server with a samplin
\item Connect the output hardware to the sound card and to the measurement equipment.
\item Use the octave skript \texttt{calibr
\end{itemize}

\subsection{Using the measured frequency response to calibrate a hearing aid output device}



\bibliography{MHA}

\printindex

\end{document}

% LocalWords:  audiogram \mha{} LTASS

%%% Local Variables: 
%%% mode: latex
%%% TeX-master: "openMHA_application_manual"
%%% indent-tabs-mode: nil
%%% coding: utf-8-unix
%%% End:
